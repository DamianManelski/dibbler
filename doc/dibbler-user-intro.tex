\section{Intro}
First of all, as an author I would like to thank you for your interest
in this DHCPv6 implementation. If this documentation doesn't answer
your question or you have any suggestions, feel free to contact
me. See \emph{Contact} section for details. Also be sure to check out
Dibbler's website located at \emph{http://klub.com.pl/dhcpv6/}.

\section{Overview}

Dibbler is a portable DHCPv6 solution. It features both server and
client. Currently there are implementations available for WindowsXP
and Linux systems. Besides basic functionality specified in RFC3315,
it also offers serveral enhancements, e.g. DNS servers and doman names
configuration.

Dibbler is an open source software, distrubuted under GNU GPL
licence. It means that it is freely available, free of charge and can
be used by anyone (including commercial usage). Sources are also
available, so anyone skilled enough can fix bugs or add new features.

As for now, Dibbler offers these features:
\begin{itemize}
\item basic server discovery and address retrival (SOLICIT,
  ADVERTISE, REQUEST and REPLY messages) -- simplest case possible:
  client discovers server, then asks for an address, which is granted
  by server.
\item discovering best server -- when client receives more than one
  ADVERTISE messages from different server, it chooses the best 
  and remembers remaining ones as a backup servers.
\item many servers support -- client is capable of discovering and
  dealing with multiple servers. For example, client would like to
  have 5 addresses configured. Prefered server can only grant 3, so
  client send request for 2 remaining addresses to one of the
  remaining servers.
\item address renewal (RENEW,REBIND and REPLY messages) -- client renews
  addresses at certain time intervals, if server specified so. 
\item duplicate address detection (DECLINE and REPLY messages) -- client
  can detect and properly handle faulty situation, when server grants
  address which is illegaly used by some other host. It will inform
  server of such circumstances, and request for another
  address. Server will mark this address as used by unknown host, and
  will assign another address to a client.
\item power failure/crash support (CONFIRM and REPLY messages) -- after
  client recovers from crash or power failure, it still can have
  assigned valid addresses. In such circumstances, client uses CONFIRM
  message, to config if those addresses are still valid. 
  \footnote{As for now, this functionality works on server side only,
  client side support is in progress.}
\item IA Option -- this option is used to carry addresses. Both server
  and client support multiple IAs in one message. Additional feature
  is client capability to ask for a specific address.
\item Rapid Commit Option (SOLICIT and REPLY messages) -- if both
  client and server are configured to use rapid commit, address
  assignment procedure can be shortened to 2 messages. It's major
  advantage is lesser network usage and quicker client startup time.
\end{itemize}

Except RFC3315-specified behavior, Dibbler also support several enhancements:

\begin{itemize}
\item DNS Servers Option -- client can ask for information about DNS
  servers. 
\item Domain Name Option -- client can ask for information about
  domain name it is connected in.
\item Time Zone Option -- client can ask for information about 
time zone it is currently in.
\item NTP Servers Option -- client can ask for Network Time Protocol
  Servers to synchronize its clock
\end{itemize}

And now some implementation specific details:
\begin{itemize}
\item On both server and client, after each modification data is
  dumped in XML format, so it can be easily processed in a automated
  manner. Simple example of advantage is a script, which can generate
  reports about server usage (assigned addresses, clients configured
  and so on).
\item Dibbler is fully portable. Core logic is system independent and
  coded in C++ language. There is about dozen low-level functions,
  which are system speficic. They're used for adding addresses,
  retriving information about interfaces, setting DNS servers and so
  on. Porting Dibbler to other systems (and even other architectures)
  would require implementic only those serveral system-specific
  functions.
\end{itemize}

See RELEASE-NOTES for details about version-specific upgrades, fixes
and features.