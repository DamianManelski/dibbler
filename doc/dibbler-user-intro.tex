%%
%% $Id: dibbler-user-intro.tex,v 1.10 2006-07-01 12:46:09 thomson Exp $
%%
%% $Log: not supported by cvs2svn $
%% Revision 1.9  2005/08/07 17:54:52  thomson
%% Minor changes related to 0.4.1 release.
%%
%% Revision 1.8  2005/07/31 15:56:11  thomson
%% WinNT/2000 port update, various small changes, typos etc.
%%
%% Revision 1.7  2005/07/21 23:28:56  thomson
%% Documentation update.
%%
%% Revision 1.6  2005/02/01 23:06:48  thomson
%% *** empty log message ***
%%
%% Revision 1.5  2005/01/23 23:16:56  thomson
%% Relay related things.
%%
%% Revision 1.4  2004/10/25 20:45:54  thomson
%% Option support, parsers rewritten. ClntIfaceMgr now handles options.
%%
%% Revision 1.3  2004/06/19 19:51:14  thomson
%% Various fixes.
%%
%% Revision 1.2  2004/06/19 10:24:59  thomson
%% Hyperlinks in PDF, building process modified
%%
%%

\section{Intro}
First of all, as an author I would like to thank you for your interest
in this DHCPv6 implementation. If this documentation doesn't answer
your question or you have any suggestions, feel free to contact
me. See \emph{Contact} section for details. Also be sure to check out
Dibbler website located at \url{http://klub.com.pl/dhcpv6/}.

%%\href{http://klub.com.pl/dhcpv6/}{Dibbler's website}.

\section{Overview}

Dibbler is a portable DHCPv6 solution. It features server, client and
relay. Currently there are ports available for Windows XP and 2003 (support for
NT4 and 2000 is considered experimental) and Linux 2.4/2.6 systems. 
It supports both stateful (i.e. IPv6 address
granting) and stateless (i.e. options granting) autoconfiguration.
Besides basic functionality\footnote{specified in RFC3315}, it also offers
serveral enhancements, e.g. DNS servers and doman names
configuration.

Dibbler is an open source software, distributed under GNU GPL
licence. It means that it is freely available, free of charge and can
be used by anyone (including commercial users). Sources are also
available, so anyone skilled enough can fix bugs, add new features and
release his/her own version.

As for now, Dibbler offers these features:
\begin{itemize}
\item Basic server discovery and address assignment (SOLICIT,
  ADVERTISE, REQUEST and REPLY messages) -- simplest case possible:
  client discovers server, then asks for an address, which is granted
  by a~server.
\item Best server discovery -- when client receives more than one
  ADVERTISE messages from different servers, it chooses the best one
  and remembers remaining ones as a backup.
\item Many servers support -- client is capable of discovering and
  dealing with multiple servers. For example, client would like to
  have 5 addresses configured. Prefered server can only grant 3, so
  client send request for remaining 2 addresses to one of the
  remaining servers.
\item Relay support -- Dibbler server supports indirect
  communication with clients via relays. Stand-alone relay implementation is also
  available. Clients can talk to the server directly or via relays.
\item Unicast communication -- if specific conditions are met, client
  could send messages directly to a server's unicast address, so
  additional servers does not need to process those messages. It also
  improves effciency, as all nodes present in LAN segment receive
  multicast packets.\footnote{Nodes, which do not belong to specific
    multicast group, drop those packets silently. However, determining
    if host belongs or not to a group must be performed on each node.}
\item Address renewal (RENEW,REBIND and REPLY messages) -- client renews
  addresses at certain time intervals, if server specified so.
\item Duplicate address detection (DECLINE and REPLY messages) -- client
  can detect and properly handle faulty situation, when server grants
  address which is illegaly used by some other host. It will inform
  server of such circumstances, and request for another
  address. Server will mark this address as used by unknown host, and
  will assign another address to a client.
\item Power failure/crash support (CONFIRM and REPLY messages) -- after
  client recovers from crash or power failure, it still can have
  assigned valid addresses. In such circumstances, client uses CONFIRM
  message, to config if those addresses are still valid%
  \footnote{As for 0.4.2 version, this functionality works on server side only,
  client side support will be available in future releases.}.
\item IA Option -- this option is used to carry addresses. Both server
  and client support multiple IAs in one message. Additional feature
  is client capability to ask for a specific address.
\item TA Option -- this option is used to carry temporary
  addresses. Both server and client supports this. Note that it is
  better to use non-temporary, normal (ia) addresses. If you don't
  know, what TA option is, you definetely don't need it.
\item Rapid Commit Option (SOLICIT and REPLY messages) -- if both
  client and server are configured to use rapid commit, address
  assignment procedure can be shortened to 2 messages. Major
  advantage is lesser network usage and quicker client startup time.
\end{itemize}

Except RFC3315-specified behavior, Dibbler also support several enhancements:

\begin{itemize}
\item DNS Servers Option -- client can ask for information about DNS
  servers. DHCPv6 server will provide those addresses.
\item Domain Name Option -- client can ask for information about
  domain name it is connected in.
\item Time Zone Option -- client can ask for information about 
time zone it is currently in.
\item NTP Servers Option -- client can ask for Network Time Protocol
  Servers to synchronize its clock.
\item SIP Servers Option -- SIP servers IPv6 address information can
  be passed to clients.
\item SIP domain name - SIP domain name can be passed to clients.
\item NIS, NIS+ server Option -- Both NIS and NIS+ server adresses
  can be passed to clients.
\item NIS, NIS+ domain name Option -- NIS or NIS+ domain names can be
  passed to clients.
\item Option renewal mechanism (Lifetime Option) -- options obtained
  from server can be updated periodically.
\item Dynamic DNS Updates -- server can assign a fully qualified
  domain name for a client. Client is able to perform DNS Update,
  i.e. inform DNS server about its name. Currently server is able to
  provide such names, but does not support the DNS Update procedure by
  itself and 
\end{itemize}

And now some implementation specific details:
\begin{itemize}
\item Server, client and relay, after each action dumps state to disk
  in a XML format, so it can be easily processed in an automated
  manner. Simple example of this advantage is a script, which can generate
  reports about server usage (assigned addresses, clients configured
  and so on);
\item Dibbler is fully portable. Core logic is system independent and
  coded in C++ language. There are also several low-level functions,
  which are system speficic. They're used for adding addresses,
  retriving information about interfaces, setting DNS servers and so
  on. Porting Dibbler to other systems (and even other architectures)
  would require implementic only those serveral system-specific
  functions.
\item Although Dibbler was developed on the i386 architecture, there
  are ports available for other architectures: IA64, AMD64, PowerPC,
  HPPA, Sparc, MIPS, S/390 and Alpha. They are available in the PLD
  Linux Distribution 2.0 as well as well as in Debian GNU/Linux. You
  can download them from \url{http://www.pld-linux.org/} or
  \url{http://www.debian.org}. Keep in mind
  that author has not tested those ports, so there might be some
  unknown issues present.
\end{itemize}

See RELEASE-NOTES for details about version-specific upgrades, fixes
and features.

\section{Requirements}
Dibbler can be run on Linux systems with kernels from 2.4 and 2.6
series. Obviously, IPv6 (compiled into kernel or as module) support is
required to run. DHCPv6 uses UDP ports below 1024, so root privileges
are required. They're also required to add, modify and delete various
system parameters, e.g. IPv6 addresses.

Dibbler also runs on Windows XP and 2003. In XP systems, at least
Service Pack 1 is required. To install various Dibbler parts (server,
client or relay) as services, administrator privileges might be
required. 

Support for Windows NT4 and 2000 is limited and considered
experimental. Due to lack of support and any kind of informations from
Microsoft, don't expect this state to change.
