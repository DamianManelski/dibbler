%%
%% Dibbler - a portable DHCPv6
%%
%% authors: Tomasz Mrugalski <thomson@klub.com.pl>
%%          Marek Senderski <msend@o2.pl>
%%
%% released under GNU GPL v2 or later licence
%%
%% $Id: dibbler-devel-intro.tex,v 1.3 2004-11-28 11:14:07 thomson Exp $
%%
%% $Log: not supported by cvs2svn $
%% Revision 1.2  2004/11/25 01:16:36  thomson
%% *** empty log message ***
%%
%% Revision 1.1  2004/07/05 01:04:40  thomson
%% Initial version.
%%
%%

\section{Intro}
Welcome to the Dibbler developer's guide. This document describes
various aspects of the compilation and installation of Dibbler server
and client. Detailed description of the internal architecture is also
provided. People with programming background can find useful
informations here. Main purpose of this document is to help
contributors to quickly know Dibbler from the inside.

This document is intenteded just as its title states -- a guide. It is
not a thorough code description. To quickly wander around classes and
methods used, see documentation generated with the Doxygen tool (open
file \verb+doc/html/index.html+).

\section{Option values}
\A DHCPv6 is a relatively new protocol and additional options are in a
specification phase. It means that until standarisation process is
over, they do not have any officially assigned numbers. Once
standarization process is over (and RFC document is released), this
option gets an official number. 

There's pretty good chance that different implementors may choose
diffrent values for those not-yet officialy accepted options. To
change those values in Dibbler, you have to modify file
misc/DHCPConst.h and recompile server or client. Make sure that you
build everything for scratch. Use \verb+make clean+ in Linux and
\verb+Clean up solution+ in Windows before you start building a new
version.
