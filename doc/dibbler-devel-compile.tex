%%
%% Dibbler - a portable DHCPv6
%%
%% authors: Tomasz Mrugalski <thomson@klub.com.pl>
%%          Marek Senderski <msend@o2.pl>
%%
%% released under GNU GPL v2 or later licence
%%
%% $Id: dibbler-devel-compile.tex,v 1.13 2006-07-31 02:10:07 thomson Exp $
%%

\section{Compilation}
Currently Dibbler supports two platforms: Linux with kernels 2.4 and
2.6 series and Windows (XP and 2003). Compilation process is system
dependent, so it is described for Linux and Windows separately.

\subsection{Linux}
To compile Dibbler, extract sources, and type:
\begin{verbatim}
make client
make server
\end{verbatim}
to build client and server. Although parser files are generated using
flex and bison++ and those generated sources are included, so there is
no need to generate them. To generate it if someone wants to generate it
by hand instead of using those supplied versions, here are appropriate
commands:
\begin{verbatim}
make parser
\end{verbatim}
to generate client, server and relay parsers.

There occassionaly might be problem with compilation, when different
flex version is installed in the system. Proper FlexLexer.h is
provided in the SrvCfgMgr and ClntCfgMgr directories.

\subsection{Windows}
To compile Dibbler under Windows, MS Visual Studio 2003 was
used. Project files are provided in the CVS and source archives.

Select project name (server-winxp or client-winxp), click properties,
choose ,,Debugging'' from ,,Configuration Properties''. Adjust ,,Command
arguments'' to match your directory.

Previous versions were also compiled using MS Visual Studio 2002, but
it is not used anymore and is not supported. If you are using MS
Visual Studio 2002, there might be a problem with
lowlevel-win32.c file compilation. Compiler might complain about
missing Ipv6IfIndex in \_IP\_ADDAPTER\_ADDRESSES structure. There is a simple way
to bypass this.  In \\
\verb+Program Files/Microsoft Visual Studio.NET/Vc7/PlatformSDK/Include/+ 
directory, there is \verb+IPTypes.h+ file. It contains structure:

\begin{verbatim}
typedef struct _IP_ADAPTER_ADDRESSES {
    union {
        ULONGLONG Alignment;
        struct {
            ULONG Length;
            DWORD IfIndex;
        };
    };
    struct _IP_ADAPTER_ADDRESSES *Next;
    PCHAR AdapterName;
    PIP_ADAPTER_UNICAST_ADDRESS FirstUnicastAddress;
    PIP_ADAPTER_ANYCAST_ADDRESS FirstAnycastAddress;
    PIP_ADAPTER_MULTICAST_ADDRESS FirstMulticastAddress;
    PIP_ADAPTER_DNS_SERVER_ADDRESS FirstDnsServerAddress;
    PWCHAR DnsSuffix;
    PWCHAR Description;
    PWCHAR FriendlyName;
    BYTE PhysicalAddress[MAX_ADAPTER_ADDRESS_LENGTH];
    DWORD PhysicalAddressLength;
    DWORD Flags;
    DWORD Mtu;
    DWORD IfType;
    IF_OPER_STATUS OperStatus;
} IP_ADAPTER_ADDRESSES, *PIP_ADAPTER_ADDRESSES;
\end{verbatim}

You should slightly modify it. Just add one additional field:
\verb+DWORD Ipv6IfIndex;+. Now it should look like this:

\begin{verbatim}
typedef struct _IP_ADAPTER_ADDRESSES {
    union {
        ULONGLONG Alignment;
        struct {
            ULONG Length;
            DWORD IfIndex;
        };
    };
    struct _IP_ADAPTER_ADDRESSES *Next;
    PCHAR AdapterName;
    PIP_ADAPTER_UNICAST_ADDRESS FirstUnicastAddress;
    PIP_ADAPTER_ANYCAST_ADDRESS FirstAnycastAddress;
    PIP_ADAPTER_MULTICAST_ADDRESS FirstMulticastAddress;
    PIP_ADAPTER_DNS_SERVER_ADDRESS FirstDnsServerAddress;
    PWCHAR DnsSuffix;
    PWCHAR Description;
    PWCHAR FriendlyName;
    BYTE PhysicalAddress[MAX_ADAPTER_ADDRESS_LENGTH];
    DWORD PhysicalAddressLength;
    DWORD Flags;
    DWORD Mtu;
    DWORD IfType;
    IF_OPER_STATUS OperStatus;
    DWORD Ipv6IfIndex;
} IP_ADAPTER_ADDRESSES, *PIP_ADAPTER_ADDRESSES;
\end{verbatim}

\subsubsection{Flex/bison under Windows}
As was mentioned before, flex and bison++ tools are not required to
successfully build Dibbler. They are only required, if changes are
made to the parsers. Lexer and Parser files (\verb+ClntLexer.*+, \verb+ClntParser.*+, \verb+SrvLexer.*+ and
\verb+SrvParser.*+) are generated by author and placed in CVS and
archives. There is no need to generate them. However, if you insist on
doing so, there is an flex and bison binary included in port-winxp. Take note that
several modifications are required:

\begin{itemize}
\item To generate \verb+ClntParser.cpp+ and \verb+ClntLexer.cpp+ files, you can use
\verb+parser.bat+. After generation, in file \verb+ClntLexer.cpp+ replace: \verb+class istream;+
with: \verb+#include <iostream>+ and \verb+using namespace std;+ lines.
\item flex binary included is slightly modified. It generates

\begin{verbatim}
#include "FlexLexer.h"
\end{verbatim} 
instead of 
\begin{verbatim}
#include <FlexLexer.h>
\end{verbatim} 

You should
add .\ to include path if you have problem with missing \verb+FlexLexer.h+.
Also note that \verb+FlexLexer.h+ is modified (std:: added in several places,
\verb+<fstream.h>+ is replaced with \verb+<fstream>+ etc.)
%%\item In file ClntParser.cpp, substitute line (around 1860): ,,	*++yyvsp = yylval;''
%%with: ,,*++yyvsp = ::yylval;''. This trick is supposed to fix numerous
%%parser problems.
\end{itemize}

Keep in mind that author is in no way a flex/bison guru and found this method
in a painful trial-and-error way. 

\subsection{DEB and RPM Packages}
There is a possibility to generate RPM (RadHat, Fedora Core, Mandrake,
PLD and lots of other distributions) and DEB (Debian, Knoppix and
other) packages. Before trying this trick, make sure that you have
required tools (rpmbuild for RPM;dpkg-deb for DEB
packages). Note that this requires root privileges. 
Package generation is done by the following commands:

\begin{verbatim}
make release-deb
make release-rpm
\end{verbatim}

\subsection{Ebuild script for Gentoo}
There is also ebuild script prepared for Gentoo users. It is located
in the Port-linux/gentoo directory. 

\subsection{Dibbler in Linux distributions}
Dibbler is available in several distributions:

\begin{description}
 \item[\href{http://debian.org}{Debian GNU/Linux}] -- use standard tools
 (apt-get, aptitude) to install dibbler-client, dibbler-server,
 dibbler-relay or dibbler-doc packages (e.g. apt-get install dibbler-client)
 \item[\href{http://www.gentoo.org}{Gentoo Linux}] -- use emerge to
 install dibbler (e.g. emerge dibbler).
 \item[\href{http://www.pld-linux.org}{PLD GNU/Linux}] -- use standard
  PLD's poldek tool to install dibbler package.
\end{description}

\subsection{Compilation environment}
When compilation is being performed in non-standard envrionment, it is a
good idea to examine and modify \verb+Makefile.inc+ file. Compiler name,
compilation and link options, used libraries and debugging options can
be modified there.

\subsection{Changing default values}
Custom builds might be prepared with different than default
compilation options. Here is a list of features, which can be
customised:
\begin{itemize}
\item Default log level -- please modify LOGMODE\_DEFAULT define in
  \verb+Mish/Logger.h+.
\item FIXME - describe remaining parameters
\end{itemize}

\subsection{Modular features}
\label{modular-features}
In the 0.5.0 release, so called \emph{modular features} were
introduced. It is now possible to enable or disable of the Dibbler
features. To set, which optional features should be compiled, modify
\verb+Makefile.inc+ file before starting compilation\footnote{In Windows
builds, which use MS Visual Studio, those flags must be defined in the
project options window}. Following flags are available:

\begin{description}
 \item[MOD\_CLNT\_EMBEDDED\_CFG] -- If this flag is set, client will use
  hardcoded configuration, instead of reading configuration file. To
  reasonably use this feature, hardcoded configuration should be
  modified to match specific needs. See \verb+ClntCfgMgr/ClntCfgMgr.cpp+
  file for details.
 \item[MOD\_CLNT\_DISABLE\_DNSUPDATE] -- If this flag is set, client will
  be compiled without DNS Update support, used in FQDN feature. This
  will make client binary file smaller and will skip the whole poslib
  library, but client will not be able to perform DNS Updates on its own
  and will ask server to perform such updates. When DNS Updates are
  disabled, extra care should be used during server configuration, so
  all updates will performed on the server side.
 \item[MOD\_CLNT\_BIND\_REUSE] -- Normally it does not make sense to
  execute server and client on the same machine. It is also not
  reasonable to execute several client instances on the same host. To
  prevent such situations, client open normal sockets (without reuse
  flag). If second client instance is executed, it will fail to create
  and bind sockets, because required address/port combination is already
  used by the first instance. However, in some situations this safety
  check can be unwanted and situation to allow to execute several
  clients in parallel should be allowed. To allow this, enable flag
  MOD\_CLNT\_BIND\_REUSE. Note that feature will also make possible to
  execute server and client on the same node.
 \item[MOD\_SRV\_DISABLE\_DNSUPDATE] -- If this flag is set, server will
  be compiled without DNS Update support, used in FQDN feature. This
  will make server binary file smaller and will skip the whole poslib
  library, but server will not be able to perform DNS Updates on client
  behalf. According to FQDN standard \cite{draft-fqdn}, only server is allowed
  to execute reverse resolve (PTR record) DNS updates, so in such setup
  only forward resoling (AAAA record) will be executed be the client.
\end{description}

