%%
%% $Id: dibbler-user-epilogue.tex,v 1.8 2006-07-16 11:24:51 thomson Exp $
%%
%% $Log: not supported by cvs2svn $
%% Revision 1.7  2006/02/02 23:36:56  thomson
%% 0.4.2 release.
%%
%% Revision 1.6  2005/12/06 19:42:27  thomson
%% Greetings updated.
%%
%% Revision 1.5  2005/07/31 15:56:11  thomson
%% WinNT/2000 port update, various small changes, typos etc.
%%
%% Revision 1.4  2004/10/25 20:45:54  thomson
%% Option support, parsers rewritten. ClntIfaceMgr now handles options.
%%
%% Revision 1.3  2004/06/19 19:51:14  thomson
%% Various fixes.
%%
%% Revision 1.2  2004/06/19 10:24:59  thomson
%% Hyperlinks in PDF, building process modified
%%
%%

\section{History}
 Dibbler project was started as master thesis by Tomasz Mrugalski and
Marek Senderski on Computer Science faculty on Gdansk University of
Technology. Both authors graduated in september 2003 and soon after
started their jobs. 

During master thesis writing, it came to my attention that there are
other DHCPv6 implementations available, but none of them has been
named properly. Refering to them was a bit
silly: ,,DHCPv6 published on sourceforge.net has better support than
DHCPv6 developed in KAME project, but our DHCPv6
implementation...''. So I have decided that this implementation should
have a name. Soon it was named Dibbler after famous CMOT
Dibbler from Discworld series by Terry Pratchett.

Sadly, Marek does not have enough free time to develop Dibbler, so his
involvement is non-existent at this time. However, that does not mean,
that this project is abandoned. It is being actively developed by
me (Tomek). Keep in mind that I work at full time and do
Ph.D. studies, so my free time is also greatly limited.

\section{Contact}
There is an website located at \url{http://klub.com.pl/dhcpv6}. If
you belive you have found a bug, please put it in Bugzilla -- it is a
bug tracking system located at \url{http://klub.com.pl/bugzilla}. If
you are not familiar with that kind of system, don't worry. After
simple registration, you will be asked for system and Dibbler version
you are using and so on. Without feedback from users, author will not
be aware of many bugs and so will not be able to fix them. That's why
users feedback is very important. You can also send bug report
directly using e-mail. Be sure to be as detailed as possible. Please
include both server and client log files, both config and xml
files. If you are familiar with tcpdump or ethereal, traffic dumps
from this programs are also great help.

If you have used Dibbler and it worked ok, this documentation answered
all you question and everything is in order (hmmm, wake up, it must be
a dream, it isn't reality:), also send a short note to author. He can
be contated at thomson(at)klub(dot)com(dot)pl (replace (at) with @ and
dot with .). Be sure to include information which country do you live
in. It's just author's curiosity to know where Dibbler is being used
or tested.

\section{Thanks and greetings}

I would like to send my thanks and greetings to various persons.
Without them, Dibbler would not be where it is today.
\begin{description}
\item[Marek Senderski] -- He's author of almost half of the Dibbler
  code. Without his efforts, Dibbler would be simple, long forgotten
  by now master thesis.
\item[Jozef Wozniak] -- My master thesis' supervisor. He allowed me to
  see DHCP in a larger scope -- as part of total automatisation process.
\item[Jacek Swiatowiak] -- He's my master thesis consultant. He guided 
  Marek and me to take first steps with DHCPv6 implementation.
\item[Ania Szulc] -- Discworld fan and a great girl, too. She's the one
  who helped me to decide how to name this yet-untitled DHCPv6 implementation.
\item[Christian Strauf] -- Without his queries and questions, Dibbler
  would be abandoned in late 2003.
\item[Bartek Gajda] -- His interest convinced me that Dibbler is worth
  the effort to develop it further.
\item[Artur Binczewski and Maciej Patelczyk] -- They both ensured that
  Dibbler is (and always will be) GNU GPL software. Open source
  community is grateful.
\item[Josep Sole] -- His mails (directly and indirectly) resulted in
  various fixes and speeded up 0.2.0 release.
\item[Sob] -- He has ported 0.4.0 back to Win2000 and NT. As a direct
  result, 0.4.1 was released for those platforms, too.
\item[Guy ''GMSoft'' Martin] -- He has provided me with access to HPPA machine,
  so I was able to squish some little/big endian bugs. He also uploaded
  ebuild to the Gentoo portage.
\item[Bartosz ''fEnio'' Fenski] -- He taught me how much work needs to be done,
  before deb packages are considered ok. It took me some time to understand that 
  more pain for the package developer means less problems for the end user. 
  Thanks to him, Dibbler is now part of the Debian GNU/Linux distribution.
\end{description}
