%%
%% Dibbler - a portable DHCPv6
%%
%% authors: Tomasz Mrugalski <thomson@klub.com.pl>
%%          Marek Senderski <msend@o2.pl>
%%
%% released under GNU GPL v2 or later licence
%%
%% $Id: dibbler-devel-misc.tex,v 1.4 2005-07-21 23:28:56 thomson Exp $
%%
%% $Log: not supported by cvs2svn $
%% Revision 1.3  2004/12/04 23:47:57  thomson
%% Work around for bug #56 described.
%%
%% Revision 1.2  2004/12/01 20:55:49  thomson
%% Documentation updated.
%%
%% Revision 1.1  2004/07/05 01:04:40  thomson
%% Initial version.
%%
%%

\section{Tips}

\begin{itemize}
\item Linux: Running client and server on the same host requires
  client recompilation with specific option enabled. Please edit
  \verb+misc/Portable.h+ and set \verb+CLIENT_BIND_REUSE+ to
  \verb+true+. This will allow to receive data from local server, but
  will also disable checking if there is another client running. So
  you can run multiple clients, which is a straight road to
  trouble. You were warned.
\item Ethereal, a widely used network sniffer/analyzer has a bug with
  parsing DHCPv6 message: SIP options are always reported as
  malformed. Also NIS/NIS+ options have improper values (not
  comformant to RFC3898). To work around that problem, download
  packet-dhcpv6.c from Dibbler homepage and recompile
  Ethereal. Dibbler's author sent patches to the Ethereal team. Those
  changes should be included in the next Ethereal release.
\item If you are reading this Developer's Guide, then Hey! You're
  probably a developer! If you found any bugs (or think you found
  one), go to the
  \href{http://klub.com.pl/bugzilla}{http://klub.com.pl/bugzilla}
   and report it. If your report was a mistake -- oh well, you just
  lost 5 minutes. But if it was really a bug, you have just helped improve
  next Dibbler version.
\item If you have any questions about Dibbler or DHCPv6, feel free to
  mail me, preferably via Dibbler mailing list. All links are provided
  on the project website.
\end{itemize}
