%%
%% Dibbler - a portable DHCPv6
%%
%% authors: Tomasz Mrugalski <thomson@klub.com.pl>
%%          Marek Senderski <msend@o2.pl>
%%
%% released under GNU GPL v2 only licence
%%
%% $Id: dibbler-devel-misc.tex,v 1.7 2008-08-29 00:07:38 thomson Exp $
%%

\section{FAQ}
This section describes various Dibbler aspects.

\begin{description}
\item[XML files] -- After performing any action, server, client and
relay store their internal information into XML files. As for 0.4.1
version, those files are never read, just written. This feature can be
used as a debugging tool. However, it's main purpose is the ability to
process and present internal state in some external form. For example
using with css styles or after processing via XSLT parsers, server
statistics can be presented as a web page.

\item[Message building] -- Each TMsg object (see Messages/Msg.h) has
Options list. Options (TOpt derived objects) are created (usually in
the constructor). They're stored as objects. For good example, see
appendRequestedOptions() method in the client messages
(ClntMessages/ClntMsg.cpp). Each option and message has method
storeSelf(), which is called just before message is being sent.

You might ask: what about retransmissions? Message is built each time
it is being resent. That might seem inefficient, but there is one
option called Estimated. It specifies how long does this particular
transaction is being processed. So each time retrasmission is in fact
a slightly different message. It differs in that option, so
UDP checksum is different, so it has to be rebuilt.
\end{description}

\section{Tips}

\begin{itemize}
\item Linux: Running client and server on the same host requires
  client recompilation with specific option enabled. Please edit
  \verb+misc/Portable.h+ and set \verb+CLIENT_BIND_REUSE+ to
  \verb+true+. This will allow to receive data from local server, but
  will also disable checking if there is another client running. So
  you can run multiple clients, which is a straight road to
  trouble. You were warned.
\item Ethereal, a widely used network sniffer/analyzer has a bug with
  parsing DHCPv6 message: SIP options are always reported as
  malformed. Also NIS/NIS+ options have improper values (not
  comformant to RFC3898). To work around that problem, download
  packet-dhcpv6.c from Dibbler homepage and recompile
  Ethereal. Dibbler's author sent patches to the Ethereal team. Those
  changes should be included in the next Ethereal
  release. \textbf{NOTE:} This is no longer true. Patch was accepted
  and now Ethereal prints informations properly.
\item If you are reading this Developer's Guide, then Hey! You're
  probably a developer! If you found any bugs (or think you found
  one), go to the
  \href{http://klub.com.pl/bugzilla}{http://klub.com.pl/bugzilla}
   and report it. If your report was a mistake -- oh well, you just
  lost 5 minutes. But if it was really a bug, you have just helped improve
  next Dibbler version.
\item If you have any questions about Dibbler or DHCPv6, feel free to
  mail me, preferably via Dibbler mailing list. All links are provided
  on the project website.
\end{itemize}
