\section{Portability Guide}

This section contains guidelines and tips for people intending to port
Dibbler to a new achitecture or system. Before attempting to do so,
please contact Dibbler author
(\href{mailto:thomson(at)klub.com.pl}{thomson(at)klub.com.pl})
for help. Substantial support will be provided.

\subsection{Low-level System API}

To port dibbler to a new system, several of the low level functions have to
be implemented. List of those functions is available in Misc/Portable.h
file, in section labeled as:

\begin{verbatim}
/* ****************************************************************** */
/* *** interface/socket low level functions ************************* */
/* ****************************************************************** */
\end{verbatim}

Here is a description of the function prototypes:

\begin{description}
\item[struct iface * if\_list\_get()] -- returns pointer to a list of iface
structures. Each structure represents a network interface. This structure is
defined in the Misc/Portable.h file. This function should allocate memory
for this list.

\item[void if\_list\_release(struct iface * list)] -- releases list previously
allocated in the if\_list\_get() function.

\item[int ipaddr\_add(const char* ifacename, int ifindex, \\const char*
	   addr, uint pref, uint valid)] -- \\
This function adds address specified (in plain text) in addr parameter to
the interface named ifacename with interface index ifindex with preferred
and valid lifetimes set to pref and valid. Note that some systems might
ignore interface name and use ifindex only, or vice versa.

\item[int ipaddr\_del(const char* ifacename, int ifindex, const char*
	   addr)] -- removes address addr (specified in plain text) from the interface
ifacename.

\item[int sock\_add(char* ifacename,int ifaceid, char* addr, int port, int
thisifaceonly, int reuse)] -- \\ create socket used to read and write data to
the ifacename/ifaceid interface, bound to address addr (specified in plain
text) and to the port. thisifaceonly parameter specifies if the socket
should be bound to the specific interface (1) or not (0). Some systems (e.g.
Linux) allow to bind socket in a way that the address/port combination can
be bound multiple times. This kind of socket binding allow some advanced
tricks like running both server and client on the same host. This parameter
is specified by MOD\_CLNT\_BIND\_REUSE, defined (or not) Makefile.inc. This
function return file descriptor used to reference to a created socket.

\item[int sock\_del(int fd)] -- delete previously created socket. fd is a
file descriptor returned by the sock\_add() function.

\item[int sock\_send(int fd, char* addr, char* buf, int buflen, int port,
int iface)] -- sends data to addr (defined in packed name)/port, using socket
fs. Send buflen byte starting at buf. Send the data using interface iface.

\item[int sock\_recv(int fd, char* myPlainAddr, char* peerPlainAddr, char*
buf, int buflen)] -- receive data from the fd socket. Store destination (my)
address in a memory located at myPlainAddr, store sender's address in a
memory located at peerPlainAddr. The data itself should be stored in a
memory located at buf. buflen is a size of a buffer (to avoid buffer
overflow). This function returns number of bytes received.

\item[int is\_addr\_tentative(char* ifacename, int iface, char* plainAddr)]
-- returns information if the address plainAddr added to the ifacename/iface
interface is tentative (1) or not (0). It is possible that the Duplicate
Address Detection is not yet complete, so other possible return value is
inconclusive (2).
\end{description}

Following functions are used to set corresponding parameters, received
from the DHCPv6, in the system:

\begin{verbatim}
int dns_add(const char* ifname, int ifindex, const char* addrPlain);
int dns_del(const char* ifname, int ifindex, const char* addrPlain);
int domain_add(const char* ifname, int ifindex, const char* domain);
int domain_del(const char* ifname, int ifindex, const char* domain);
int ntp_add(const char* ifname, int ifindex, const char* addrPlain);
int ntp_del(const char* ifname, int ifindex, const char* addrPlain);
int timezone_set(const char* ifname, int ifindex, const char*timezone);
int timezone_del(const char* ifname, int ifindex, const char*timezone);
int sipserver_add(const char* ifname, int ifindex, const char*addrPlain);
int sipserver_del(const char* ifname, int ifindex, const char*addrPlain);
int sipdomain_add(const char* ifname, int ifindex, const char*domain);
int sipdomain_del(const char* ifname, int ifindex, const char*domain);
int nisserver_add(const char* ifname, int ifindex, const char*addrPlain);
int nisserver_del(const char* ifname, int ifindex, const char*addrPlain);
int nisdomain_set(const char* ifname, int ifindex, const char*domain);
int nisdomain_del(const char* ifname, int ifindex, const char*domain);
int nisplusserver_add(const char* ifname, int ifindex, const char*addrPlain);
int nisplusserver_del(const char* ifname, int ifindex, const char*addrPlain);
int nisplusdomain_set(const char* ifname, int ifindex, const char*domain);
int nisplusdomain_del(const char* ifname, int ifindex, const char*domain);
\end{verbatim}

There are also inet\_pton4() (IPv4 address Plain-To-Network), inet\_pton6
(IPv6 address Plain-To-Network), inet\_ntop4 (IPv4 address Network-To-Plain)
and inet\_ntop6 (IPv6 address Network-To-Plain) functions, which should be
present in the system. If they are not, port-specific part of the dibbler
should provide them.

Also function microsleep(int x) should make current process dormant for x
microseconds.

An example implementation of those functions, can be found in
Port-linux/layer3.c and Port-linux/lowlevel-options-linux.c file. Those
files are specific for a Linux system.

To fully port Dibbler, also a main() function must be implemented. It should
contain system-specific interface (e.g. registration as a service in Windows
environment or detaching to background in Linux "daemon" mode).
It is also necessary to include following code in the client implementation:

\begin{verbatim}
    TDHCPClient client(CLNTCONF\_FILE);
    client.run();
\end{verbatim}

Where CLNTCONF\_FILE is a filename of a client configuration file. Similar
code should be executed in the server implementation:

\begin{verbatim}
    TDHCPServer srv(SRVCONF\_FILE);
    srv.run();
\end{verbatim}

See Port-linux/dibbler-client.cpp and Port-linux/dibbler-server.cpp for
example implementation, specific to a Linux systems. Implementations for
Windows XP are available in the Port-win32 directory.
