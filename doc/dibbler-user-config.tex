
\section{Config files}

This section describes Dibbler server and (optional) client
configuration. Square brackets denotes optional values: mandatory
[optional]. Alternative is marked as $\mid$. A $\mid$ B means A or B.

\subsection{Tokens and basic informations}
Config file parsing is token-based. Here's list of tokens used:
\begin{description}
\item[IPv6 address] -- IPv6 address 
\item[32-bit decimal integer] -- string containing only numbers, e.g. 123456
\item[32-bit hex integer] -- string constaining only numbers and
  letter a--f, ended with h letter, e.g. 23afh
\item[string] -- string of arbitrary characters enclosed in single
  quotes, e.g. 'this is string'
\item[DUID identifier] -- hex number enclosed in double quotes, e.g. ''deadbeaf''
\end{description}

There are also some client or server specified tokens. Here's the list:
\begin{description}
\item[client] -- iface, no-config, address, ia, no-ia,
log-level, work-dir, append, prepend, require, request, send, default,
supersede, prefered-lifetime, valid-lifetime, t1, t2, dns-servers,
domain, ntp-servers, time-zone, reject-servers, prefered-servers,
rapid-commit 
\item[server] -- iface, no-config, class, log-level, work-dir,
  dns-servers, domain, ntp-server, time-zone, accept-only,
  reject-only, t1, t2, prefered lifetime, valid-lifetime, unicast,
  preference, pool, rapid-commit, max-lease, client-max-lease 
\end{description}

\subsection{Scopes}
There are four scopes, in which options can be specified:
\begin{itemize}
\item global
\item inteface
\item IA
\item address
\end{itemize}

Global scope is the largest. It covers the whole config file and
applies to all intefaces, IAs, and addresses, until some lower level
options override it. Next comes inteface scope. Options defined there
are inteface-specific and apply to this interface, all IAs in this
interface and addresses in those IAs. Next is IA scope. Options
defined there are IA-specific and apply to this IA and to addresses it
contains. Least significant scope is address. Every option is specific
for one scope. For example, T1 is defined for
IA scope. However, it can be also used in more common scopes. In this
case -- in interface or global. Defining T1 in interface scope means:
,,for this interface default value for T1 is ...''. The same applies
to global scope. Options can be used multiple times. In that case
value defined later is used.

\subsection{Comments}

Comments are also allowed. All common comment styles are supported:
\begin{itemize}
\item C++ style one-line comments: // this is comment
\item C style multi-line comments: /* this is multiline comment */
\item bash style one-line comments: \# this is one-line comment
\end{itemize}

Parses is case-insensitive, so Iface, IfAcE, iface and IFACE mean the
same. This does not apply to interface names, of course. eth0 and ETH0
are dwo diffrent interfaces.

\subsection{Client config file}

Client config file should be named \emph{client.conf}. After
successful startup, old version of this file is stored as
\emph{client.conf-old}. One of design requirements for client was
,,out of the box'' usage. To achieve this, simply use empty
client.conf file. Client will try to get one address for each up and
running interface \footnote{Exactly: Client tries to configure each up
  and running interface, which has link address at lease 6 bytes
  long. So it will not configure tunnels (which usually have IPv4
  address (4bytes long) as their link address. It should configure all
  Ethernet and 802.11 interfaces. The latter was not tested by author,
  because of lack of access to 802.11 equipment.}.

\subsubsection{Global scope}

Every option can be declared in global scope.
Config file has this form:

\begin{verbatim}
 [interface declaration	 |
  global options         |
  interface options      |
  IA options             |
  address options        ]
\end{verbatim}

\subsubsection{Interface declaration}

Interface can be declared this way:
\begin{verbatim}
iface name_of_this_interface
{
  interface options      |
  IA options             |
  address options        
}
\end{verbatim}

or 

\begin{verbatim}
iface number 
{
  interface options      |
  IA options             |
  address options        
}
\end{verbatim}

or
\begin{verbatim}
iface name_of_this_interface no-config
\end{verbatim}

or 
\begin{verbatim}
iface number no-config
\end{verbatim}

In every case, number denotes interface number. It can be extracted
from ,,ip~l'' (Linux) or ,,ipv6~if'' (Windows). \opt{no-config} is
depreciated and means that this interface should not be
configured. Client retrives interface list from the system. 

Also take a note that name of the interface no longer needs to be
enclosed in single or double quotes. It is necessary only in windows
systems, where interface names sometimes contains spaces.

If interface contains no IA keyword, one IA with one address is
assumed. If interface should not request for address, \opt{no-ia}
should be used or this interface can be omited in config file.

\subsubsection{IA declaration}

IA is declared this way:

\begin{verbatim}
ia [number]
[{ 
  [address declaration   | 
   IA options            | 
   address options       ] 
   ...          
}]
\end{verbatim}

where number is optional number, which describes how many such IAs
should be requested. If this number is not equal 1, then address
options are not allowed. That could come in handy when someone need
serveral IAs with the same parameters. If IA contains no addresses,
client assumes that one address should be configured.


\subsubsection{Address declaration}
Addres is declared like this:

\begin{verbatim}
address [number] 
[{ 
	[address options   |
	IPv6 address    ] 
}]
\end{verbatim}
where number denotes how many addresses with those values should be
requested. If it is diffrent than 1, then IPv6 address option is not
allowed.

\subsubsection{Standard options}
Standard options are... well, standard. This means that they have 
nothing to do with any extensions. Standard options are declared this way:

\begin{verbatim}
OptionName option-value
\end{verbatim}

Every option has a scope it can be used in, default value and
sometimes allowed range.

\begin{tabular}{|c|c|p{2.5cm}|p{3cm}|p{6cm}|}
\hline
Name           & Scope & Values & default & Description \\
               &       & (default) &    & \\
\hline
valid-lifetime    & address& 32-bit integer & (4294967296) & valid lifetime for address (specified in seconds) (H)\\
prefered-lifetime&address& 32-bit integer& (4294967296) & after this amount of time(in seconds) address becomes depreciated (H)\\
T1               & IA     & 32-bit integer & (4294967296) & client should renew addresses after T1 seconds (H)\\
T2               & IA     & 32-bit integer & (4294967296) & client should send REBIND after T2 seconds (H)\\
reject-servers   & IA     & IPv6 addrs or DUID list, coma separated &
(null list)      & list containing servers which should be discarded in configuration of this IA \\
prefered-servers & IA     & IPv6 addrs or DUID list & (null list) & Prefered servers list. ADVERTISE messages received by client are sorted according to this list. \\
rapid-commit     & IA     & 0 or 1         & 0            & should we use Rapid Commit? \\
work-dir         & global & string & (null) & working directory \\
log-level        & global & 1-8    & 8    & log-level (8 is most verbose) \\
log-name         & global & string & Client & Name, which appears in a log file\\
log-mode         & global & short | full & full & logging mode: short (date
and name suppressed) or full\footnote{In future versions, syslog (Linux) and
  eventlog (Windows) is also planned)} \\
\hline
\end{tabular}

\subsubsection{Extension options}
Extension options are options specified in external drafts and (in a
near future) in RFC documents. They are declared with ,,option''
keyword:

\begin{verbatim}
option OptionName option-value
\end{verbatim}

where OptionName is one of possible values listed below:

\begin{tabular}{|c|c|p{2.5cm}|p{3cm}|p{6cm}|}
\hline
OptionName     & Scope & Values    & default & Description \\
               &       & (default) &         & \\
\hline
dns-servers    & interface& IPv6 addrs list, coma separated & (null list) & prefered DNS servers list (H) \\
domain         & interface& domain & (null) & prefered domain (H)\\
ntp-servers    & interface& IPv6 addrs list, coma separated & (null list) & prefered NTP servers list (H)\\
time-zone      & interface& timezone string & (null) & time zone (H)\\

\hline
\end{tabular}

Note that timezone format is described in file draft-ietf-dhc-dhcpv6-opt-tz-00.txt 
and domain format is described in RFC 3646.

\subsubsection{Client Example}

In simplest case, client config can be empty. Client will try to
assign one address for every interface present in the system, except
interfaces:
\begin{itemize}
\item which are down (flag UP not set)
\item loopback (flag LOOPBACK set)
\item which are not running (flag RUNNING not set)
\item which are not multicast capable (flag MULTICAST not set)
\end{itemize}

If you must use DHCPv6 on one of such interfaces (which is not
recommended and probably will fail), you must specify this interface
in config file.

Let's consider more coplicated case. Let's say there are 4 interfaces,
numbered 1 thru 4. Interfaces 1,2 and 3 are not to be
configured. Interface 4, named eth0 should have 3 IAs. Two of them are
supposed to contain one address each. Third IA should contain 3
addresses. Addresses assigned to first and second IA should have
prefered-lifetime 1 hour and valid-lifetime 2 hours. This IA should
have 3 specific addresses: 2000::1, 2000::2 and 2000::3. Information
about NTP servers,our current timezone, available DNS servers and our
domain should also be retrived,

Here's config file:

\begin{verbatim}
iface eth0
{
  valid-lifetime 7200    // default value - 2 hours
  prefered-lifetime 3600 // default value - 1 hour
  T1 600                 // request 10 minutes interval
  T2 1200		 /* trouble begins after 20 minutes
                            of server's silence */
  IA 2                   // 2 IAs with just specified values (1h/2h/10min/20min)
  IA			 // third IA is more specific
  {
    valid-lifetime 3600   	// valid lifetime changed to 1 hour
    prefered-lifetime 1800	// prefered lifetime changed to 30min
    address
    {			  
      2000::1		  // request those addresses
      2000::2             
      2000::3
    }
  }
  option ntp-servers       // ask for NTP servers
  option time-zone         // ask for timezone
  option dns-servers       // ask for DNS servers
  option domain            // ask for domain
}
\end{verbatim}

\subsection{Server config file}

Client config file should be named \emph{server.conf}. After
successful startup, old version of this file is stored as \emph{server.conf-old}.

\subsubsection{Global scope}

Every option can be declared in global scope.
Config file has this form:

\begin{verbatim}
 interface declaration	|
 global options         |
 interface options      |
 class options          
\end{verbatim}

\subsubsection{Interface declaration}

Interface can be declared this way:
\begin{verbatim}
iface name_of_this_interface
{
  interface options      |
  class options        
}
\end{verbatim}

or 

\begin{verbatim}
iface number 
{
  interface options      |
  class options        
}
\end{verbatim}

where name\_of\_this\_interface denotes name of the interface and
number denotes it's number. It no longer needs to be enclosed in
single or double quotes (except windows cases, when interface name
contains spaces).

\subsubsection{Class scope}
Address class is declared as follows:

\begin{verbatim}
class
{  
     class options |
     address poll    
}
\end{verbatim}

address poll has this format:
\begin{verbatim}
poll minaddress-maxaddress
\end{verbatim}

\subsubsection{Options}

Every option has a scope it can be used in, default value and
sometimes allowed range.

\begin{tabular}{|c|c|p{2.5cm}|p{3cm}|p{6cm}|}
\hline
Name           & Scope & Values & default & Description \\
               &       & (default) &  & \\
\hline
valid-lifetime & class& 32-bit integer & (4294967296) & valid lifetime for address (specified in seconds)\\
prefered-lifetime&class& 32-bit integer& (4294967296) & after this amount of time(in seconds) address becomes depreciated\\
T1             & class & 32-bit integer & (4294967296) & client should renew addresses after T1 seconds \\
T2             & class & 32-bit integer & (4294967296) & client should send REBIND after T2 seconds\\
reject-clients & class & IPv6 addrs or DUID list, coma separated & (null list) & list containing servers which should be discarded in configuration of this IA \\
accept-only    & class & IPv6 addrs or DUID list & (null list) & these are the only clients allowed to use this class\\
RapidCommit    & class & 0 or 1         & 0            & should we allow Rapid Commit (SOLICIT--REPLY)? \\
work-dir       & global & string & (null) & working directory \\
log-level      & global & 1-8    & 8    & log-level (8 is most verbose) \\
log-name       & global & string & Client & Name, which appears in a log file\\
log-mode       & global & short | full & full & logging mode: short (date
and name suppressed) or full\footnote{In future versions, syslog (Linux) and
  eventlog (Windows) is also planned)} \\
dns-servers    & interface& IPv6 addrs list, coma separated & (null list) & DNS servers list \\
domain         & interface& domain & (null) & domain name \\
ntp-servers    & interface& IPv6 addrs list, coma separated & (null list) & NTP servers list \\
time-zone      & interface& timezone string & (null) & time zone \\
unicast	       & class & 0 or 1         & 0 & is unicast communication allowed? \\
preference     & class & 0-255          & 0 & server preference value (higher is more prefered) \\
max-lease      & class & 32-bit integer & 4294967296 & how many addresses can be leased by clients? \\
client-max-lease & class & 32-bit integer & 4294967296 & how many addresses can be leased by one client? \\
\hline
\end{tabular}

\subsubsection{Server example}

In opposite to client, server uses only interfaces described in config
file. Examine this common situation: server has interface
named \emph{eth0} (which is fourth interface in the system) and is
supposed to assign addresses from 2000::100/124 class. Simplest config
file looks like that:

\begin{verbatim}
iface eth0
{ 
  class
  {
    pool 2000::100-2000::10f
  } 
}
\end{verbatim}

Let's extend this with numerous requirements:
\begin{itemize}
\item add new class on fifth interface named \emph{eth1}, for example 2000::fe00/120, for
  which Rapit commit will be allowed
\item server preference set to maximum (255)
\item assign 2000::20/124 class on fourth interface named
  eth0. Preference value for this interface should be 0.
\item don't ignore client with DUID ``00001231200adeaaa'' in the class
  2000::20/124.
\item valid and prefered lifetimes are 1 hour and 30 minutes
  respectively. T1 and T2 set to 10 minutes and 20 minutes.
\item For class 2000::100/124 valid and prefered lifetimes are 2 hours
  and 1 hour.
\item in class 2000::100/124 one client can request up to two
  addresses
\item do not assign more than 10 addresses from 2000::20/124 class.
\item client with fe80::200:39ff:fe4b:1abc address should get his
  static address 2000::2f
\item we shall support DNS and NTP servers on interface 5. Timezone
  and domains information is also supported.
\item working directory and log level is also set
\end{itemize}

Here's config to do all this stuff:

\begin{verbatim}
log-level 5
work-dir '/var/lib/dibbler'
valid-lifetime 3600
prefered-lifetime 1800
T1 600
T2 1200
iface 4
{
  preference 0
  class
  {
    reject-clients ``00001231200adeaaa''
    2000::2f-2000::20  // it's in reverse order, but it works.
                       // just a trick. 
    max-lease 10
  }
  class
  {
    accept-only fe80::200:39ff:fe4b:1abc
    pool 2000::2f 
  }
}
iface 5
{
  dns-server 2000::123:456,2000::456:1234
  ntp-server 2000::1111:2222
  rapid-commit 1
  time-zone ''EST''
  domain ''example.com''
  preference 255
  class
  {
    pool 2000::fe00-2000::feff
  }
  
  class
  {
    valid-lifetime 7200
    prefered-lifetime 3600
    pool 2000::0-2000::f
    client-max-lease 2
  } 
}
\end{verbatim}
