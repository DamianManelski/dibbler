%%
%% Dibbler - a portable DHCPv6
%%
%% authors: Tomasz Mrugalski <thomson@klub.com.pl>
%%
%%
%% released under GNU GPL v2 licence
%%
%% $Id: dibbler-user-config.tex,v 1.35 2006-12-31 00:29:21 thomson Exp $
%%

\section{Configuration files}

This section describes Dibbler server, relay  and client
configuration. Square brackets denotes optional values: mandatory
[optional]. Alternative is marked as $\mid$. A $\mid$ B means A or
B. Parsers are case-insensitive, so Iface, IfAcE, iface and IFACE mean
the same. This does not apply to interface names. eth0 and
ETH0 are dwo diffrent interfaces.

\subsection{Data types}
Config file parsing is token-based. Token can be considered a keyword
or a specific phrase. Here's list of tokens used:
\begin{description}
\item[IPv6 address] -- IPv6 address, e.g. 2000:dead:beef::789
\item[32-bit decimal integer] -- string containing only numbers, e.g. 123456
\item[string] -- string of arbitrary characters enclosed in single or double
  quotes, e.g. 'this is a string'. If string contains only a-z, A-Z and
  0-9 characters, quotes can be omited, e.g. beeblebrox
\item[DUID identifier] -- hex number starting with 0x, e.g. 0x12abcd.
\item[IPv6 address list] -- IPv6 addresses separated with commas,
	   e.g. 2000::123, 2000::456
\item[DUID list] -- DUIDs separated with commas, e.g. 0x0123456,0x0789abcd
\item[string list] -- strings separated with comas, e.g. tealc,jackson,carter,oneill
\item[boolean] -- YES, NO, TRUE, FALSE, 0 or 1. Each of them can be
  used, when user must enable or disable specific option.
\end{description}

\subsection{Scopes}
There are four scopes, in which options can be specified: global,
inteface, IA and address. Every option is specific for one scope.
Each option is only applied to a scope and all subscopes in which it is
defined.

For example, T1 is defined for
IA scope. However, it can be also used in more common scopes. In this
case -- in interface or global. Defining T1 in interface scope means:
,,for this interface the default T1 value is ...''. The same applies
to global scope. Options can be used multiple times. In that case
value defined later is used.

Global scope is the largest. It covers the whole config file and
applies to all intefaces, IAs, and addresses, unless some lower scope
options override it. Next comes inteface scope. Options defined there
are inteface-specific and apply to this interface, all IAs in this
interface and addresses in those IAs. Next is IA scope. Options
defined there are IA-specific and apply to this IA and to addresses it
contains. Least significant scope is address. 

\subsection{Comments}

Comments are also allowed. All common comment styles are supported:
\begin{itemize}
\item C++ style one-line comments: // this is comment
\item C style multi-line comments: /* this is multiline comment */
\item bash style one-line comments: \# this is one-line comment
\end{itemize}

\subsection{Client configuration file}
\label{client-cfg-file}
Client configuration file should be named \verb+client.conf+. It should be
placed in the \verb+/etc/dibbler/+ directory (Linux system) or in the
current directory (Windows systems). After successful startup, old
version of this file is stored as \verb+client.conf-old+. One of
design requirements for client was ,,out of the box'' usage. To
achieve this, simply use empty 
\verb+client.conf+ file. Client will try to get one address for each up and
running interface \footnote{Exactly: Client tries to configure each
  up, multicast-capable and running interface, which has link address
  at least 6 bytes long. So it will not configure tunnels (which
  usually have IPv4 address (4bytes long) as their link address. It
  should configure all Ethernet and 802.11 interfaces. The latter was
  not tested by author due to lack of access to 802.11 equipment.}.

\subsubsection{Global parameters}

There are several global options. Those options can be used in the
global scope. However, inteface, IA and address scoped options can be
used in the global scope, too. 
Client configuration file has following syntax:

\begin{verbatim}
global-options
interface-options
IA-options
address-options
interface-declaration
\end{verbatim}

\subsubsection{Interface declaration}

Each system interface, which should be configured, must be mentioned in
the configureation file. Interfaces can be declared with this syntax:
\begin{verbatim}
iface interface-name
{
  interface-options
  IA-options
  address-options        
}
\end{verbatim}

or 

\begin{verbatim}
iface interface-number 
{
  interface-options
  IA-options
  address-options        
}
\end{verbatim}

In the latter case, interface-number denotes interface number. It can be extracted
from ,,ip~l'' (Linux) or
,,ipv6~if'' (Windows). \verb+interface-name+ is an interface
name.  Name of the interface does not have to be enclosed in single or
double quotes. It is necessary only in Windows systems, where interface
names sometimes contain spaces, e.g. ''local network connection''.
Interface scoped options can be used here. IA-scoped as well as address
scoped options can also be used. They will be treated as a default
values for future definitions of the IA and address instantations.

\subsubsection{IA declaration}
IA is an acronym for Identity Association. It is a logical entity
representing address or addresses used to perform some
functions. IA-options can be defined, e.g. T1. IPv6 addresses can be
defined here. All those values will be used as hints for a server.
Almost always, each DHCPv6 client will have exactly one IA on each
interface. IA is declared using following syntax:

\begin{verbatim}
ia
{ 
  IA-options
  address-options
  address-declaration
}
\end{verbatim}

It is also possible to define multiple IA at once. To do so, following
syntax might be used:

\begin{verbatim}
ia number
{ 
  IA-options
  address-options
}
\end{verbatim}
Number is an optional number, which describes how many such IAs
should be requested. Number is optional. If it is not specified, 1 is
used. If this number is not equal 1, then address options are not
allowed. That could come in handy when someone need serveral IAs with
the same parameters. If IA contains no addresses, client assumes that
one address should be configured. IA scoped as well as address options
can be defined here. IA scoped options will be applied directly, while
address scoped options will be used as default values for all addresses
that will be defined in this IA. 

\subsubsection{Address declaration}
When IA is defined, it is sometimes useful to define its address. Its
value will be used as a hint for the server. Address is declared in the
following way:

\begin{verbatim}
address number
{ 
  address-options
  IPv6-address
}
\end{verbatim}
where number denotes how many addresses with those values should be
requested. If it is diffrent than 1, then IPv6 address options are not
allowed. Only address scoped options can be used here.

\subsubsection{Standard options}
So called standard options are defined by the base DHCPv6
specification, a so called RFC 3315 document \cite{rfc3315}. Those options are
called standard, because all DHCPv6 implementations, should properly
handle them. Standard options are declared in the following way:

\begin{verbatim}
OptionName option-value
\end{verbatim}

Every option has a scope it can be used in, default value and
sometimes allowed range. 

\begin{description}
 \item[work-dir] -- (scope: global, type: string, default: .) Defines working
	    directory.
 \item[log-level] -- (scope: global, type: integer, default: 7) Defines
	    verbose level of the log messages. The valid range if
	    from 1 (Emergency) to 8 (Debug). The higher the logging
	    level is set, the more messages dibbler will print.
 \item[log-name] -- (scope: global, type: string, default: Client). Defines 
	    name, which should be used during logging.
 \item[log-mode] -- (scope: global, type: short, full or precise,
	    default value: full) Defines logging mode. In the
	    default, full mode, name, date and time in the h:m:s format
	    will be printed. In short mode, only minutes and
	    seconds will be printed (this mode is useful on
	    terminals with limited width). Recently added precise
	    mode logs information with seconds and microsecond
	    precision. It is a useful for finding bottlenecks in
	    the DHCPv6 autoconfiguration process.
 \item[strict-rfc-no-routing] -- (scope: global, type: none, default:
	    not defined). During normal operation, DHCPv6 client
	    should add IPv6 address only, without configuring
	    routing, because this should be done with other means,
	    i.e. router advertisements \cite{rfc2461}. However,
	    this behavior is confusing and lots of users complained
	    about it, so since the 0.5.0-RC1 release, this has been changed
	    in dibbler. Right now when dibbler client configures
	    address, it also configures routing, so every host is
	    able to communicate with other hosts, which have
	    obtained address from the same server. If you don't
	    like this behavior, you might want to use this option.
 \item[duid-type] -- (scope: global, type: DUID-LLT, DUID-LL or DUID-EN,
	    default: DUID-LLT). This parameter defines, what type of
	    DUID should be generated if there is no DUID already
	    present. If there is a file containing DUID, this directive
	    has no effect. DUID-LLT means that DUID will be based on
	    link layer address as well as time. DUID-LL means that only
	    link layer address will be used. The last value -- DUID-EN
	    -- Enterprise Number-based generation is not currently
	    supported.
 \item[rapid-commit] -- (scope: interface, type: boolean, default:
	    0). This option allows rapid commit procedure to be
	    performed. Note that enabling rapid commit on the client
	    side is not enough. server must be configured to allow
	    rapid commit, too.
 \item[unicast] -- (scope: interface, type: boolean, default: 0). This
	    option specifies if client should request unicast
	    communication from the server. If server is configured to
	    allow it, it will add unicast option to its replies. It will
	    allow client to communicate with server via unicast
	    addresses instead of usual multicast.
 \item[prefered-servers] -- (scope: interface, type: address or duid list, default:
	    empty). This list defines, which servers are prefered. When
	    client sends \msg{SOLICIT} message, all servers available in
	    the local network will respond. When client receives
	    multiple \msg{ADVERTISE} messages, it will choose those sent
	    by servers mentioned on the perfered-server list.
 \item[reject-servers] -- (scope: interface, type: address or duid list, default:
	    empty) This list defines which server must be ignored. It
	    has negative meaning to the prefered-servers list.
 \item[vendor-spec] -- (scope: interface, type: integer-hexstring,
	    default: empty). This option allow requesting for a vendor
	    specific configuration option. It does not any good in
	    itself as there are no dibbler-specific options to
	    configure. It can be, however, used to test some other
	    DHCPv6 server implementations. In short words: if you don't
	    know what that is, you don't need it.
 \item[T1] -- (scope: IA, type: integer: default: $2^{32}-1$). This value
	    defines after what time client should start renew
	    process. This is only a hint for the server. Actual value
	    will be provided by the server.
 \item[T2] -- (scope: IA, type: integer, default:$2^{32}-1$). This value
	    defines after what time client will start emergency rebind
	    procedure if renew process fails. This is only a hint for
	    the server. Actual value will be provided by the server.
 \item[valid-lifetime] -- (scope:address, type: integer,
	    default:$2^{32}-1$) This parameter defines valid lifetime of
	    an address. It will be used as a hint for a server, when the
	    client will send requests.
 \item[prefered-lifetime] -- (scope:address, type: integer,
	    default:$2^{32}-1$) This parameter defines prefered lifetime
	    of an address. It will be used as a hint for a server, when
	    there client will send requests.
\end{description}

\subsubsection{Extension options}
Extension options are the options specified in external drafts and RFC
documents, but not in the base spec \cite{rfc3315}. To easily
distinguish if an option is part of the base standard or one of the
multiple extensions, \verb+option+ keyword was added in the extension
options declaration. Therfore extension options are declared as follows:

\begin{verbatim}
option option-name
\end{verbatim}

or

\begin{verbatim}
option option-name option-value
\end{verbatim}

where option-name is name of the options. First approach instructs
dibbler client to just ask for this particular option. Second approach
includes requested values. When sent by the client, server will use
those values as hints during those options assignment. Since those
options are defined per interface, thus every extension option has an
interface scope, i.e. it is defined once per interface. As for the
\version release, currently supported options are:

\begin{description}
 \item[dns-server] -- (scope: interface, type: address list, default:
	    none). This option conveys information about DNS servers
	    available. After retriving this information, client will be
	    able to resolve domain names into IP (both IPv4 and IPv6)
	    addresses. Without setting up DNS servers, host's network
	    capability is greatly reduced, as user can't use domain
	    names (e.g. http://wp.pl/), but must use IP addresses
	    directly (e.g. http://212.77.100.101/ or
	    http://3ffe:1234::456/). Defined in \cite{rfc3596}.
 \item[domain] -- (scope: interface, type: domain list, default:
	    none). This option is used for retriving domain or domains
	    names, which the client is connected in. For example, if
	    client's hostname is \verb+alice.mylab.example.com+ and it wants to
	    contact \verb+bob.mylab.example.com+ it can simply refer to it as
	    \verb+bob+. Without domain name configured, it would have to
	    use full domain name. After successful configuration, this
	    useful shortcut is being used by all services available: web
	    browsing, mail sending, news reading etc. Defined in
	    \cite{rfc3596}.
 \item[ntp-server] -- (scope: interface, type: address list, default:
	    none). This option defines information about available NTP
	    servers. Network Time Protocol \cite{rfc2030} is a protocol used
	    for time synchronisation, so all hosts in the network has
	    the same proper time set. Defined in \cite{rfc4075}.
 \item[time-zone] -- (scope: interface, type: timezone, default:
	    none). It is possible to retrieve timezone from the
	    server. If client is interested in this information, it
	    should ask for this option. Note that this option is
	    considered obsolete as it is mentioned in draft version only
	    \cite{draft-timezone}. Work on this draft seems to be
	    abandoned as similar functionality is provided in now
	    standard \cite{rfc4075}.
 \item[sip-server] -- (scope: interface, type: address list, default:
	    none). Session Initiation Protocol \cite{rfc3263} is an
	    control protocol for creating, modifying, and terminating
	    sessions with one or more participants. These sessions
	    include Internet telephone calls, multimedia distribution,
	    and multimedia conferences. Its most common usage is
	    VoIP. Format of this option is defined in \cite{rfc3319}.
 \item[sip-domain] -- (scope: interface, type: domain list, default:
	    none). It is possible to define domain names for Session
	    Initiation Protocol \cite{rfc3263}. Configuration of this
	    parameter will ease usage of domain names in the SIP
	    protocol. Format of this option is defined in
	    \cite{rfc3319}.
 \item[nis-server] -- (scope: interface, type: address list, default:
	    none). Network Information Service (NIS) is a Unix-based
	    system designed to use common login and user information on
	    multiple systems, e.g. universities, where students can log
	    on to ther accounts from any host. To use this
	    functionality, a host needs information about NIS server's
	    address. This can be retrieved with this option. Its format
	    is defined in \cite{rfc3898}.
 \item[nis-domain] -- (scope: interface, type: domain list, default:
	    none). Network Information Service (NIS) can albo specify
	    domain names. It can be configured with this option. It is
	    defined in \cite{rfc3898}.
 \item[nis+-server] -- (scope: interface, type: address list, default:
	    none). Network Information Service Plus (NIS+) is an
	    improved version of the NIS protocol. This option is defined
	    in \cite{rfc3898}.
 \item[nis+-domain] -- (scope: interface, type: domain list, default:
	    none). Similar to nis-domain, it defines domains for
	    NIS+. This option is defined in \cite{rfc3898}.
 \item[lifetime] -- (scope: interface, type: boolean, default: no). Base
	    spec of the DHCPv6 protocol does offers way of refreshing
	    addresses only, but not the options. Lifetime defines, how
	    often client would like to renew all its options. By default
	    client will not send such option, but it will accept it and
	    act accordingly if the server sends it on its own. Format of
	    this option is defined in \cite{draft-lifetime}.
\end{description}

Note that timezone format is described in file \verb+draft-ietf-dhc-dhcpv6-opt-tz-00.txt+
and domain format is described in RFC 3646. After receiving options
values from a server, client stores values of those options in separate
files in the working directory (\verb+/var/lib/dibbler+ in Linux and
current directory in Windows). File names start with the option word,
e.g. \verb+option-dns-server+. Several options are also processed and
set up in the system. Options supported in Linux and Windows
environments are presented in the table below.


\begin{center}
\begin{tabular}{|l|l|l|l|}
\hline
Option & Linux & WinXP/2003 & WinNT/2000  \\
\hline
dns-server  & system, file & system, file & system,file \\
domain      & file         & file & file \\
ntp-server  & file         & file & file \\
time-zone   & file         & file & file \\
sip-server  & file         & file & file \\
sip-domain  & file         & file & file \\
nis-server  & file         & file & file \\
nis-domain  & file         & file & file \\
nis+-server & file         & file & file \\
nis+-domain & file         & file & file \\
\hline
\end{tabular}
\end{center}

\subsubsection{Stateless configuration}

If interface does not contain \verb+IA+ or \verb+TA+ keywords, client
will ask for one address (one IA with one address request will be sent).
If client should not request any addresses on this interface,
\opt{stateless}\footnote{In the version 0.2.1-RC1 and earlier, this
  directive was called no-ia. This depreciated name is valid for now,
  but might be removed in future releases.} keyword must be used. In
such circumstances, only specified options will be requested.

\subsubsection{Relay support}
Usage of the relays is not visible from the client's point of view:
Client can't detect if it communicates via relay(s) or directly
with the server. Therefore no special directives on the client side 
are required to use relays. 

\subsection{Client configuration examples}
This subsection contains various examples of the most popular
configurations. Several additional examples are provided with the source
code. Please download it and look at \verb+*.conf+ files.

\subsubsection{Example 1: Default}
In the most simple case, client configuration file can be empty. Client will try to
assign one address for every interface present in the system, except
interfaces, which are:
\begin{itemize}
\item down (flag UP not set)
\item loopback (flag LOOPBACK set)
\item not running (flag RUNNING not set)
\item not multicast capable (flag MULTICAST not set)
\item have link-layer address less than 6 bytes long (this requirement
      should skip all tunnels and virtual interfaces)
\end{itemize}

If you must use DHCPv6 on one of such interfaces (which is not
recommended and such attempt probably will fail), you must explicitly
specify this interface in the configuration file.

\subsubsection{Example 2: DNS}
Configuration mentioned in previous subsection is a minimal one and in a
real life will be used rarely. The most common usage of the DHCPv6
protocol is to request for an address and DNS configuration. Client
configuration file achieving those goals is presented below:
\begin{Verbatim}
# client.conf
log-mode short
log-level 7
iface eth0 {
  ia
  option dns-server
}
\end{Verbatim}

\subsubsection{Example 3: Timeouts and specific address}

Automatic configuration is being driven by several timers, which define,
what action should be performed at various intervals. Since all
values are provided by the server, client can only define values, which
will be sent to a server as hints. Server might take them into
consideration, but might also ignore them
completely. Following example shows how to ask for a specific address
and provide hints for a server. Client would like to get 2000::1:2:3
address, it would like to renew addresses once in 30 minutes (T1 timer
is set to 1800 seconds). Client also would like to have address, which
is prefered for an hour and is valid for 2 hours. 

\begin{Verbatim}
# client.conf
log-mode short
log-level 7
iface eth0 {
  T1 1800
  T2 2000
  prefered-lifetime 3600
  valid-lifetime 7200
  ia {
    address { 
      2000::1:2:3
    }
  }
}
\end{Verbatim}

\subsubsection{Example 4: Unicast, more than one address}

Another example: client like to obtain 2 addresses on
,,Local Area Connection'' interface. Note quotation marks around
interface name. They are necessary since this particular interface name
contains spaces. Client also would like to accept Unicast
communication if server supports it. User don't care for
details, so keep those log very short. Take note that you won't be
able see to what Dibbler is doing with such low log-level. (Usually
log-level should be set to 7, which is also a default value).

\begin{Verbatim}
# client.conf
log-mode short
log-level 5
iface "Local Area Connection" {
  unicast yes
  ia 2
}
\end{Verbatim}

\subsubsection{Example 5: Quick configuration using Rapid-commit}
Rapid-commit is a shortened exchange with server. It consists of only
two messages, instead of the usual four. It is worth to know that both sides (client
and server) must also support rapid-commit to use this fast
configuration. 

\begin{Verbatim}
# client.conf
iface eth1 {
  rapid-commit yes
  ia 
  option dns-server
}
\end{Verbatim}

\subsubsection{Example 6: Stateless mode}
Client can be configured to work in a stateless mode. It means that it
will obtain only some configuration parameters, but no
addresses. Let's assume we want all the details stored in a log file and
we want to obtain all possible configuration parameters. Here is a
configuration file:

\begin{Verbatim}
# client.conf
log-level 8
log-mode full
stateless
iface eth0
{
  option dns-server
  option domain
  option ntp-server
  option time-zone
  option sip-server
  option sip-domain
  option nis-server
  option nis-domain
  option nis+-server
  option nis+-domain
}
\end{Verbatim}

\subsubsection{Example 7: Dynamic DNS (FQDN)}
\label{example-client-fqdn}
Dibbler client is able to request fully qualified domain name,
i.e. name, which is fully resolvable using DNS. After receiving such
name, it can perform DNS Update procedure. Client can ask for any
name, without any preferrence. Here is an example how to configure
client to perform such task:
\begin{Verbatim}
# client.conf
log-level 7
iface eth0 {
# ask for address
    ia

# ask for options
   option dns-server
   option domain
   option fqdn
}
\end{Verbatim}

In this case, client will mention that it is interested in FQDN by
using Option Request and empty FQDN option, as specified in
\cite{draft-fqdn}. Server upon receiving such request (if it is
configured to support it), will provide FQDN option containing domain
name. Depending on the server's configuration, all DNS Updates will be
performed by the server, forward will be performed by client and reverse
by the server, or only forward will be done by a client.

It is also possible for client to provide its name as a hint for
server. Server might take it into consideration when it will choose a
name for this client. Example of a configuration file for such
configuration is provided below:

\begin{Verbatim}
# client.conf
log-level 7
iface eth0 {
    # ask for address
    ia

    # ask for options
    option dns-server
    option domain
    option fqdn zoe.example.com
}
\end{Verbatim}

Note that to successfully perform DNS Update, address must be assigned
and dns server address must be known. So ,,ia'' and ,,option
dns-server'' are required for ,,option fqdn'' to work properly. Also if
DHCPv6 server provides more than one DNS server address, update will
be attempted only forthe first address on the list.


\subsubsection{Example 8: Interface indexes}
Usually, interface names are referred to by names, e.g. eth0 or Local
Area Connection. Every system also provides unique number associated
with each infterface, usually called ifindex or interface index. It is
possible to read the number using \verb+ip l+ command (Linux) or
\verb+ipv6 ifx+. Below is an example, which demonstrate how to use
interface indexes:

\begin{Verbatim}
# client.conf
log-mode short
log-level 5
iface 5 {
  ia
}
\end{Verbatim}

\subsubsection{Example 9: Vendor-specific options}
\label{example-client-vendor-spec}
It is possible to configure dibbler-client to ask for a vendor specific
options. Although there are no dibbler-specific features to configure,
it is possible to use this option to test other server
implementations. This option will rather be used by network engineers
and power network admins, rather than normal end users. 

There are 3 ways to define, how dibbler-client can request
vendor-specific options. First choice: It can just ask for this option (only 
\opt{option request option} will be sent). Second choice: it can ask for
vendor-spec option by adding such option with enterprise number set, but
no actual data. Third choice: send this option and include both
enterprise number and actual data. In the following configuration file
example, uncomment appropriate line to obtain desired bahavior:

\begin{Verbatim}
# client.conf
log-level 8
iface eth0 {
# ask for address
    ia

# uncomment only one of the following lines:
   option vendor-spec
#   option vendor-spec 1234
#   option vendor-spec 1234 0x0002abcd
}
\end{Verbatim}

\subsection{Server configuration file}

Server configuration is stored in \verb+server.conf+ file in the
+/etc/dibbler+ (Linux systems) or in current (Windows systems)
directory. 

\subsubsection{Global scope}

Every option can be declared in a global scope. Global options can be
defined here. Also options of a smaller scopes can be defined here --
they will be used as a default values. Configuration file has following syntax:

\begin{verbatim}
 global-options
 interface-options
 class-options          
 interface-declaration
\end{verbatim}

\subsubsection{Interface declaration}
Each network interface, which should be serviced by the server, must be
mentioned in the configuration file. Network interface is defined like this:
\begin{verbatim}
iface interface-name
{
  interface-options
  class-options        
}
\end{verbatim}

or 

\begin{verbatim}
iface number 
{
  interface-options
  class-options        
}
\end{verbatim}

where \verb+interface-name+ denotes name of the interface and
\verb+interface-number+ denotes its number. Name no longer needs to be
enclosed in single or double quotes (except Windows systems, when
interface name contains spaces). Note that virtual interfaces, used
to setup relay support are also declared in this way.

\subsubsection{Class scope}
Class is a smallest scope used in the server configuration file. It
contains definition of the addresses, which will be provided to
clients. Only class scoped parameters can be defined here. Address class
is declared as follows:
\begin{verbatim}
class
{  
  class-options
  address-pool    
}
\end{verbatim}

Address pool defines range of the addresses, which can be assigned to the
clients. It can be defined in one of the following formats:
\begin{verbatim}
pool minaddress-maxaddress
pool address/prefix
\end{verbatim}

\subsubsection{Standard options}

So called standard options are defined by the base DHCPv6 specification,
a so called RFC 3315 document \cite{rfc3315}. Those options are
called standard, because all DHCPv6 implementations, should properly
handle them. Each option has a specific scope it belongs to. 

Standard options are declared in the following way:

\begin{verbatim}
OptionName option-value
\end{verbatim}

\begin{description}
 \item[work-dir] -- (scope: global, type: string, default: .) Defines working
	    directory.
 \item[log-level] -- (scope: global, type: integer, default: 7) Defines
	    verbose level of the log messages. The valid range if
	    from 1 (Emergency) to 8 (Debug). The higher the logging
	    level is set, the more messages dibbler will print.
 \item[log-name] -- (scope: global, type: string, default: Server). Defines 
	    name, which should be used during logging.
 \item[log-mode] -- (scope: global, type: short, full or precise,
	    default value: full) Defines logging mode. In the
	    default, full mode, name, date and time in the h:m:s format
	    will be printed. In short mode, only minutes and
	    seconds will be printed (this mode is useful on
	    terminals with limited width). Recently added precise
	    mode logs information with seconds and microsecond
	    precision. It is a useful for finding bottlenecks in
	    the DHCPv6 autoconfiguration process.
 \item[cache-size] -- (scope: global, type: integer, default:
	    1048576). It defines a size of the memory (specified in
            bytes) which can se used to store cached entries.
 \item[preference] -- (scope: interface, type: 0-255, default:
	    none). Eech server can be configured to a specific
	    preference level. When client receives several
	    \msg{ADVERTISE} messages, it should choose that server,
	    which has the highest preference level. It is also worth
	    noting that client, upon reception of the \msg{ADVERTISE}
	    message with preference set to 255 should skip wait phase
	    for possible other \msg{ADVERTISE} messages.
 \item[unicast] -- (scope: interface, type: address,
	    default:none). Normally clients sends data to a well known
	    multicast address. This is easy to achieve, but it wastes
	    network resources as all nodes in the network must process
	    such messages and also network load is increased. To prevent
	    this, server might be configured to inform clients about its
	    unicast address, so clients, which accept it, will switch to
	    a unicast communication.
 \item[rapid-commit] -- (scope: interface, type: boolean, default:
            0). This option allows rapid commit procedure to be
            performed. Note that enabling rapid commit on the server
            side is not enough. Client must be configured to allow
	    rapid commit, too.
\item[iface-max-lease] -- (scope: interface, type: integer, default:
            $2^{32}-1$). This parameter defines, how many normal
            addresses can be granted on this interface.
\item[client-max-lease] -- (scope: interface, type: interger,
            default:$2^{32}-1$). This parameter defines, how many
            addresses one client can get. Main purpose of this
            parameter is to limit number of used addresses by
            misbehaving (malicious or restarting) clients.
\item[relay] -- (scope: interface, type: string, default: not
            defined). Used in relay definition.
            It specifies name of the physical (or name of
            another relay, if cascade relaying is used) interface,
            which is used to receive and transmit relayed data. See
            \ref{features-relays}, \ref{example-server-relay1} and
            \ref{example-server-relay2} for details.
\item[interface-id] -- (scope: interface, type: integer, default: not
            defined). Used in relay definition. Each relay interface
            should have defined its unique identified. It will be sent
            in the \opt{interface-id} option. Note that this value
            must be the same as configured in the dibbler-relay. See
            \ref{features-relays}, \ref{example-server-relay1} and
            \ref{example-relay} for details.
 \item[vendor-spec] -- (scope: interface, type: integer-hexstring,
	    default: not defined). This parameter can be used to
	    configure some vendor-specific information option. Since
	    there are no dibbler-specific options, this implementation
	    is flexible. User can specify in the configuration file,
	    how should this option look like. See
	    \ref{example-server-vendor-spec} section for details. It
            is uncommon, but possible to define several vendor
            specific options for different vendors. In such case,
            administrator must specify coma separated list. Each list
            entry is a vendor (enterprise number), ,,--'' sign and a
            hex dump (similar to DUID).
 \item[T1] -- (scope: class, type: integer or integer range: default:
            $2^{32}-1$). This value
	    defines after what time client should start renew
	    process. Exact value or accepted range can be
            specified. When exact value is defined, client's hints are
            ignored completely. 
 \item[T2] -- (scope: class, type: integer or integer range, default:$2^{32}-1$). This value
	    defines after what time client will start emergency rebind
	    procedure if renew process fails. Exact value or accepted range can be
            specified. When exact value is defined, client's hints are
            ignored completely.
\item[valid-lifetime] (scope: class, type: integer or integer range,
	    default:$2^{32}-1$). This parameter defines valid lifetime of
	    the granted addresses. If range is specified, client's
            hints from that range are accepted.
\item[prefered-lifetime] (scope: class, type: integer or integer range,
	    default:$2^{32}-1$). This parameter defines prefered
            lifetime of the granted addresses. If range is specified,
            client's hits from that range will be accepted.
\item[class-max-lease]  -- (scope: interface, type: interger,
            default:$2^{32}-1$). This parameter defines, how many
            addresses can be assigned from that class.
\item[reject-clients] -- (scope: class, type: address or DUID list,
            default: none). This parameter is sometimes called
            black-list. It is a list of a clients, which should not be
            supported. Clients can be identified by theirs link-local
            addresses or DUIDs.
\item[accept-only] -- (scope: class, type: address or DUID list,
            default: none). This parameter is sometimes called
            white-list. It is a list of supported clients. When this
            list is not defined, by default all clients (except
            mentioned in reject-clients) are supported. When
            accept-only list is defined, only client from that list
            will be supported.
\end{description}

\subsubsection{Addional options}
Server supports additonal options, not specified in \cite{rfc3315}. They have
following generic form:

\begin{verbatim}
option OptionName OptionsValue
\end{verbatim}

All supported options are specified below:

\begin{description}
 \item[dns-server] -- (scope: interface, type: address list, default:
	    none). This option conveys information about DNS servers
	    available. After retriving this information, clients will be
	    able to resolve domain names into IP (both IPv4 and IPv6)
	    addresses. Defined in \cite{rfc3596}.
 \item[domain] -- (scope: interface, type: domain list, default:
	    none). This option is used for configuring one or more domain 
	    names, which clients are connected in. For example, if
	    client's hostname is \verb+alice.mylab.example.com+ and it wants to
	    contact \verb+bob.mylab.example.com+, it can simply refer to it as
	    \verb+bob+. Without domain name configured, it would have to
	    use full domain name. Defined in \cite{rfc3596}.
 \item[ntp-server] -- (scope: interface, type: address list, default:
	    none). This option defines information about available NTP
	    servers. Network Time Protocol \cite{rfc2030} is a protocol used
	    for time synchronisation, so all hosts in the network has
	    the same proper time set. Defined in \cite{rfc4075}.
 \item[time-zone] -- (scope: interface, type: timezone, default:
	    none). It is possible to configure timezone, which is
            provided by the server. Note that this option is
	    considered obsolete as it is mentioned in draft version only
	    \cite{draft-timezone}. Work on this draft seems to be
	    abandoned as similar functionality is provided by now
	    standard \cite{rfc4075}.
 \item[sip-server] -- (scope: interface, type: address list, default:
	    none). Session Initiation Protocol \cite{rfc3263} is an
	    control protocol for creating, modifying, and terminating
	    sessions with one or more participants. These sessions
	    include Internet telephone calls, multimedia distribution,
	    and multimedia conferences. Its most common usage is
	    VoIP. Format of this option is defined in \cite{rfc3319}.
 \item[sip-domain] -- (scope: interface, type: domain list, default:
	    none). It is possible to define domain names for Session
	    Initiation Protocol \cite{rfc3263}. Configuration of this
	    parameter will ease usage of domain names in the SIP
	    protocol. Format of this option is defined in
	    \cite{rfc3319}.
 \item[nis-server] -- (scope: interface, type: address list, default:
	    none). Network Information Service (NIS) is a Unix-based
	    system designed to use common login and user information on
	    multiple systems, e.g. universities, where students can log
	    on to ther accounts from any host. Its format is defined
            in \cite{rfc3898}.
 \item[nis-domain] -- (scope: interface, type: domain list, default:
	    none). Network Information Service (NIS) can albo specify
	    domain names. It can be configured with this option. It is
	    defined in \cite{rfc3898}.
 \item[nis+-server] -- (scope: interface, type: address list, default:
	    none). Network Information Service Plus (NIS+) is an
	    improved version of the NIS protocol. This option is defined
	    in \cite{rfc3898}.
 \item[nis+-domain] -- (scope: interface, type: domain list, default:
	    none). Similar to nis-domain, it defines domains for
	    NIS+. This option is defined in \cite{rfc3898}.
 \item[lifetime] -- (scope: interface, type: boolean, default: no). Base
	    spec of the DHCPv6 protocol does offers way of refreshing
	    addresses only, but not the options. Lifetime defines, how
	    often client should renew all its options. When defined,
            lifetime option will be appended to all replies, which
            server sends to a client. If client does not support it,
            it should ignore this option. Format of
	    this option is defined in \cite{draft-lifetime}.
\end{description}

Lifetime is a special case. It is not set up by client in a system
configuration. It is, however, used by the client to know how long
obtained values are correct.

\subsection{Server configuration examples}

This subsection contains various examples of the server
configuration. If you are interested in additional examples, download source version
and look at \verb+*.conf+ files.


\subsubsection{Example 1: Simple}

In opposite to client, server uses only interfaces described in config
file. Let's examine this common situation: server has interface
named \emph{eth0} (which is fourth interface in the system) and is
supposed to assign addresses from 2000::100/124 class. Simplest config
file looks like that:

\begin{Verbatim}
# server.conf
iface eth0
{ 
  class
  {
    pool 2000::100-2000::10f
  } 
}
\end{Verbatim}

\subsubsection{Example 2: Timeouts}
Server should be configured to deliver specific timer values to the
clients. This example shows how to instruct client to renew (T1 timer) 
addresses one in 10 minutes. In case of problems, ask other servers in
15 minutes (T2 timer), that allowe prefered lifetime range is from 30
minutes to 2 hours, and valid lifetime is from 1 hour to 1 day. DNS
server parameter is also provided. Lifetime option is used to make
clients renew all non-address related options renew once in 2 hours.

\begin{Verbatim}
# server.conf
iface eth0 
{
  T1 600
  T2 900
  prefered-lifetime 1800-3600
  valid-lifetime 3600-86400
  class
  {
    pool 2000::100/80
  } 

  option dns-server 2000::1234
  option lifetime 7200
}
\end{Verbatim}

\subsubsection{Example 3: Limiting amount of addresses}
Another example: Server should support 2000::0/120 class on eth0
interface. It should not allow any client to obtain more than 5
addresses and should not grant more then 50 addresses in total. From
this specific class only 20 addresses can be assigned. Server
preference should be set to 7. This means that this server is more
important than all server with preference set to 6 or less. 
Config file is presented below:

\begin{Verbatim}
# server.conf
iface eth0
{
  iface-max-lease 50
  client-max-lease 5
  preference 7
  class
  {
    class-max-lease 20
    pool 2000::1-2000::100
  }
}  
\end{Verbatim}

\subsubsection{Example 4: Unicast communication}
Here's modified previous example. Instead of specified limits, unicast
communication should be supported and server should listen on
2000::1234 address. Note that default multicast address is still
supported. You must have this unicast address already configured on 
server's interface.

\begin{Verbatim}
# server.conf
log-level 7
iface eth0
{
  unicast 2000::1234
  class
  {
    pool 2000::1-2000::100
  }
}  
\end{Verbatim}

\subsubsection{Example 5: Rapid-commit}
This configuration can be called quick. Rapid-commit is a way to shorten exchange to only two messages. It is
quite useful in networks with heavy load. In case if client does not
support rapid-commit, another trick is used. Preference is set to
maximum possible value. 255 has a special meaning: it makes client to
skip wait phase for possible advertise messages from other servers and
quickly request addresses.

\begin{Verbatim}
# server.conf
log-level 7
iface eth0
{
  rapid-commit yes
  preference 255
  class
  {
    pool 2000::1/112
  }
}  
\end{Verbatim}

\subsubsection{Example 6: Access control}
Administrators can selectively allow certain client to use this
server (white-list). On the other hand, some clients could be
explicitly forbidden to use this server (black-list). Specific DUIDs,
DUID ranges, link-local addresses or the whole address ranges are
supported. Here is config file:

\begin{Verbatim}
# server.conf
iface eth0
{
  class
  {
    # duid of the rejected client
    reject-clients ``00001231200adeaaa''
    2000::2f-2000::20  // it's in reverse order, but it works.
                       // just a trick. 
  }
}
iface eth1
{
  class
  {
    accept-only fe80::200:39ff:fe4b:1abc
    pool 2000::fe00-2000::feff
  }
}
\end{Verbatim}

\subsubsection{Example 7: Multiple classes}
Although this is not common, a few users have requested support for multiple classes on one interface.
Dibbler server can be configured to use several classes. When client asks for an address, one of the classes
is being choosen on a random basis. If not specified otherwise, all classes have equal probability of being chosen.
However, this behavior can be modified using \verb+share+ parameter. In the following example, server supports
3 classes with different preference level: class 1 has 100, class 2 has 200 and class 3 has 300. This means that class 1
gets $\frac{100}{100+200+300} \approx 16\% $ of all requests, class 2
gets $\frac{200}{100+200+300} \approx 33\% $ and class 3 gets the rest 
($\frac{300}{100+200+300}=50\% $).

\begin{Verbatim}
# server.conf
log-level 7
log-mode short

iface eth0 {
 T1 1000
 T2 2000

 class {
   share 100
   pool 4000::1/80
 }
 class {
   share 200
   pool 2000::1-2000::ff
 }

 class {
   share 300
   pool 3000::1234:5678/112
 }
}
\end{Verbatim}

\subsubsection{Example 8: Relay support}
\label{example-server-relay1}
To get more informations about relay configuration, see section \ref{features-relays}.
Following server configuration example explains how to use
relays. There is some remote relay with will send encapsulated data over
eth1 interface. It is configured to append interface-id option set to
5020 value. Let's allow all clients using this relay some addresses
and information about DNS servers:

\begin{Verbatim}
# server.conf
iface relay1 {
  relay eth1
  interface-id 5020
  class {
    pool 2000::1-2000::ff
  }
  option dns-server 2000::100,2000::101
}
\end{Verbatim}

\subsubsection{Example 9: 2 relays}
\label{example-server-relay2}
This is advanced configuration. It assumes that client sends data to
relay1, which encapsulates it and forwards it to relay2, which
eventually sends it to the server (after additional encapsulation). It
assumes that first relay adds interface-id option set to 6011 and
second one adds similar option set to 6021. 

\begin{Verbatim}
# server.conf
iface relay1
{
  relay eth0
  interface-id 6011
} 

iface relay2
{
  relay relay1
  interface-id 6021
  T1 1000
  T2 2000
  class {
    pool 6020::20-6020::ff
  }
}
\end{Verbatim}

\subsubsection{Example 10: Dynamic DNS (FQDN)}
\label{example-server-fqdn}

Support for Dynamid DNS Updates was added recently. To configure it
on the server side, list of available names must be defined. Each name
can be reserved for a certain address or DUID. When no reservation is
specified, it will available to everyone, i.e. the first client asks
for FQDN will get this name. In following example, name 'zebuline.example.com' is
reserved for address 2000::1, kael.example.com is reserved for 2000::2 and
test.example.com is reserved for client using DUID
00:01:00:00:43:ce:25:b4:00:13:d4:02:4b:f5. 

Also note that is required to define, which side can perform updates.
This is done using single number after ,,option fqdn'' phrase. Server
can perform two kinds of DNS Updates: AAAA (forward resolving,
i.e. name to address) and PTR (reverse resolving, i.e. address to
name). To configure server to execute both updates, specify 2. This is
a default behavior. If this value will be skipped, server will attempt
to perform both updates. When 1 will be specified, server will update
PTR record only and will leave updating AAAA record to the
client. When this value is set to 0, server will not perform any
updates.

The last parameter (64 in the following example) is a prefix length of
the reverse domain supported by the DNS server, i.e. if this is set to
64, and 2000::/64 addresses are used, DNS server must support
0.0.0.0.0.0.0.0.0.0.0.0.0.0.2.ip6.arpa. zone.

\begin{Verbatim}
# server.conf
log-level 8
log-mode precise
iface "eth1" {
 prefered-lifetime 3600
 valid-lifetime 7200
 class {
   pool 2000::1-2000::ff
 }

 option dns-server 2000::100,2000::101
 option domain example.com, test1.example.com
 option fqdn 2 64
        zebuline.example.com - 2000::1,
	kael.example.com - 2000::2,
	test.example.com - 0x0001000043ce25b40013d4024bf5,
	zoe.example.com,
	malcolm.example.com,
	kaylee.example.com,
	jayne.example.com
}
\end{Verbatim}

\subsubsection{Example 11: Vendor-specific Information option}
\label{example-server-vendor-spec}
It is possible to configure dibbler-server to provide vendor-specific
information options. Since there are no dibbler-specific parameters,
this implementation is quite flexible. Enterprise number as well as
content of the option itself can be configured. 

\begin{Verbatim}
# server.conf
log-level 8
log-mode precise
iface "eth1" {
 class {
   pool 2000::1-2000::ff
 }

 option vendor-spec 1234-0x00002fedc
}
\end{Verbatim}

In some rare cases, several different options for different vendors
may be specifed. In the folloging example 2 different values are
defined, depending on which vendor client will specify in \msg{SOLICIT} or
\msg{REQUEST} message. If client will only mention that it is interested in
any vendor specific into (i.e. did not sent \opt{vendor-spec info} option, but
only mentioned in in \opt{option request} option, it will receive
first vendor option defined (in the following example, that would be a
1234 and 0002fedc).

\begin{Verbatim}
# server.conf
log-level 8
log-mode precise
iface "eth1" {
 class {
   pool 2000::1-2000::ff
 }

 option vendor-spec 1234-0x00002fedc,5678-0x0002aaaa
}
\end{Verbatim}


\subsection{Relay configuration file}

Relay configuration is stored in \verb+relay.conf+ file in the
\verb+/etc/dibbler/+ directory (Linux systems) or in current directory
(Windows systems).

\subsubsection{Global scope}

Every option can be declared in global scope.
Config file consists of global options and one or more inteface
definitions. Note that reasonable minimum is 2 interfaces, as defining
only one would mean to resend messages on the same interface.

\subsubsection{Interface declaration}

Interface can be declared this way:
\begin{verbatim}
iface name_of_the_interface
{
  interface options
}
\end{verbatim}

or 

\begin{verbatim}
iface number 
{
  interface options
}
\end{verbatim}

where name\_of\_the\_interface denotes name of the interface and
number denotes it's number. It does not need to be enclosed in
single or double quotes (except windows cases, when interface name
contains spaces).

\subsubsection{Options}

Every option has a scope it can be used in, default value and
sometimes allowed range.

%% FIXME: this table must be sanitized
\begin{tabular}{|c|c|>{\centering}p{1.7cm}<{}|c|p{6cm}|}
\hline
Name             & Scope   & Values      & default    & Description \\
                 &         & (default)   &  & \\
\hline
%%work-dir         & global  & string      & empty      & working directory \\
log-level        & global  & 1-8         & 8          & log-level (8 is most verbose) \\
log-name         & global  & string      & Relay      & Name, which appears in a log file\\
log-mode         & global  &short or full& full       & logging mode: short (date and name suppressed) or full \\

client multicast &interface& boolean     &            & Client's messages should be received on the multicast address.\\
client unicast   &interface& address     &not defined & Client's messages should be received on the specified multicast address. \\
server multicast &interface& boolean     &            & Forwarded messages should be sent to the multicast address. \\
server unicast   &interface& address     &not defined & Forwarded messages should be send to the specified address. \\
interface-id     &interface& integer     &not defined & Identifier of that particular interface. Used for interface-id option. \\
\hline
\end{tabular}

\vspace{0.5cm}

It is worth mentioning that interface-id should be specified on the
interface, which is used to receive messages from the clients, not the
one used to forward packets to server.

\subsection{Relay configuration examples}
\label{example-relay}

Relay configuration file is fairly simple. Relay forwards DHCPv6
messages between interfaces. Messages from client are encapsulated and
forwarded as RELAY\_FORW messages. Replies from server are received as
RELAY\_REPL message. After decapsulation, they are being sent back to
clients. 

It is vital to inform server, where this relayed message was
received. DHCPv6 does this using interface-id option. This identifier
must be unique. Otherwise relays will get confused when they will
receive reply from server. Note that this id does not need to be
alligned with system interface id (ifindex). Think about it as
"ethernet segment identifier" if you are using Ethernet network or as
"bss identifier" if you are using 802.11 network.

If you are interested in additional examples, download source version
and look at \verb+*.conf+ files.

\subsubsection{Example 1: Unicast/multicast}
Let's assume this case: relay has 2 interfaces: eth0 and
eth1. Clients are located on the eth1 network. Relay should receive
data on that interface using well-known ALL\_DHCP\_RELAYS\_AND\_SERVER
multicast address (ff02::1:2). Relay also listens on its global
address 2000::123. Packets received on the eth1 should be forwarded on
the eth0 interface, also using multicast address:

\begin{Verbatim}
# relay.conf
log-level 8
log-mode short
iface eth0 {
  server multicast yes
}
iface eth1 {
  client multicast yes
  client unicast 2000::123
  interface-id 1000
}
\end{Verbatim}

\subsubsection{Example 2: Multiple interfaces}
Here is another example. This time messages should be forwarded from
eth1 and eth3 to the eth0 interface (using multicast) and to the eth2
interface (using server's global address 2000::546). Also clients must
use multicasts (the default approach):

\begin{Verbatim}
# relay.conf
iface eth0 {
  server multicast yes
}
iface eth2 {
  server unicast 2000::456
}
iface eth1 {
  client multicast yes                    
  interface-id 1000
}
iface eth3 {
  client multicast yes                    
  interface-id 1001
}
\end{Verbatim}

\subsubsection{Example 3: 2 relays}
Those two configuration files correspond to the ,,2 relays'' example
provided in the server example 8. See \ref{features-relays} for details.

\begin{Verbatim}
# relay.conf - relay 1
log-level 8
log-mode full

# messages will be forwarded on this interface using multicast
iface eth2 {
   server multicast yes    // relay messages on this interface to ff05::1:3
 # server unicast 6000::10 // relay messages on this interface to this global address
}

iface eth1 {
#  client multicast yes    // bind ff02::1:2
  client unicast 6011::1   // bind this address
  interface-id 6011
}
\end{Verbatim}

\begin{Verbatim}
# relay.conf - relay 2
iface eth0 {
#   server multicast yes  // relay messages on this interface to ff05::1:3
  server unicast 6011::1  // relay messages on this interface to this global address
}

# client can send messages to multicast 
# (or specific link-local addr) on this link
iface eth1 {
  client multicast yes    // bind ff02::1:2
# client unicast 6021::1  // bind this address
  interface-id 6021
}
\end{Verbatim}
