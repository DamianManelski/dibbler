%%
%% $Id: dibbler-user-faq.tex,v 1.6 2004-12-04 23:47:57 thomson Exp $
%%
%% $Log: not supported by cvs2svn $
%% Revision 1.5  2004/12/01 21:58:43  thomson
%% *** empty log message ***
%%
%% Revision 1.4  2004/11/28 11:14:07  thomson
%% RFCs and drafts added, clarification about option values
%%
%% Revision 1.3  2004/10/25 20:45:54  thomson
%% Option support, parsers rewritten. ClntIfaceMgr now handles options.
%%
%% Revision 1.2  2004/06/19 10:24:59  thomson
%% Hyperlinks in PDF, building process modified
%%


\section{Frequently Asked Question}

Soon after Dibbler was published, I started to receive questions from
users. Some of them were common enough to get into this section.

\subsection{Common}

\Q Why client does not configure routing after assigning addresses, so
I cannot e.g. ping other hosts?

\A It's rather difficult problem. DHCP's job is to obtain address and
it exactly does that. To ping any other host, routing should be 
configured. And this should be done using Router Advertisements. It's
kinda odd, but that's the way it was meant to work. If there will be
requests from users, I'll think about some enchancements.

\Q Dibbler sends some options which have values not recognized by the
Ethereal or by other implementations. What's wrong?

\A DHCPv6 is a relatively new protocol and additional options are in a
specification phase. It means that until standarisation process is
over, they do not have any officially assigned numbers. Once
standarization process is over (and RFC document is released), this
option gets an official number. 

There's pretty good chance that different implementors may choose
diffrent values for those not-yet officialy accepted options. To
change those values in Dibbler, you have to modify file
misc/DHCPConst.h and recompile server or client. See Developer's
Guide, section \emph{Option Values} for details.

Currently options with assigned values are: RFC3315: \opt{CLIENT\_ID},
\opt{SERVER\_ID}, \opt{IA\_NA}, \opt{IAADDR}, \opt{OPTION\_REQUEST},
\opt{PREFERENCE}, \opt{ELAPSED}, \opt{STATUS\_CODE},
\opt{RAPID-COMMIT}, \opt{IA\_TA}, \opt{RELAY\_MSG}, \opt{AUTH\_MSG},
\opt{USER\_CLASS}, \opt{VENDOR\_CLASS}, \opt{VENDOR\_OPTS},
\opt{INTERFACE\_ID}, \opt{RECONF\_MSG}, \opt{RECONF\_ACCEPT}; RFC3319:
\opt{SIP\_SERVERS}, \opt{SIP\_DOMAINS}; RFC3646: \opt{DNS\_RESOLVERS},
\opt{DOMAIN\_LIST}; RFC3633: \opt{IA\_PD}, \opt{IA\_PREFIX}; RFC3898:
\opt{NIS\_SERVERS}, \opt{NIS+\_SERVERS}, \opt{NIS\_DOMAIN},
\opt{NIS+\_DOMAIN}. Take note that Dibbler does not support all of
them.

There are several options which currently does not have values
assigned (in parenthesis are numbers used in Dibbler):
\opt{NTP\_SERVERS} (40), \opt{TIME\_ZONE} (41), \opt{LIFETIME} (42),
\opt{FQDN} (43).

\subsection{Linux specific}

\Q I can't run client and server on the same host. What's wrong?

\A First of all, running client and server on the same host is just
plain meaningless, except testing purposes only. There is a problem
with sockets binding. To work around this problem, consult Developer's
Guide, Tip section how to compile Dibbler with certain options.

\Q After enabling unicast communication, my client fails to send
REQUEST messages. What's wrong?

\A This is a problem with certain kernels. My limited test capabilites
allowed me to conclude that there's problem with 2.4.20
kernel. Everything works fine with 2.6.0 with USAGI patches. Patched 
kernels with enhanced IPv6 support can be downloaded from
\url{http://www.linux-ipv6.org/}. Please let me know if your kernel
works or not.

\subsection{Windows specific}

\Q After installing \emph{Advanced Networking Pack} or \emph{Windows XP
  ServicePack2} my DHCPv6 (or other IPv6 application) stopped
   working. Is Dibbler XP SP2 compatible?

\A In both of this products, there is a IPv6 firewall installed. It
is configured by default to reject all incoming IPv6 traffic. You have
to disable this firewall. To do so, issue following commands in a
console:

\begin{verbatim}

\end{verbatim}

