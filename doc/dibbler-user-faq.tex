%%
%% $Id: dibbler-user-faq.tex,v 1.3 2004-10-25 20:45:54 thomson Exp $
%%
%% $Log: not supported by cvs2svn $
%% Revision 1.2  2004/06/19 10:24:59  thomson
%% Hyperlinks in PDF, building process modified
%%


\section{Frequently Asked Question}

Soon after Dibbler was published, I started to receive questions from
users. Some of them were common enough to get into this section.

\subsection{Common}

\Q Why client does not configure routing after assigning addresses, so
I cannot e.g. ping other hosts?

\A It's rather difficult problem. DHCP's job is to obtain address and
it exactly does that. To ping any other host, there should be routing
specified. And this should be done using Router Advertisements. It's
kinda odd, but that's the way it was meant to work. If there will be
requests from users, I'll think about some enchancements.

\subsection{Linux specific}

\Q After enabling unicast communication, my client fails to send
REQUEST messages. What's wrong?

\A This is a problem with certain kernels. My limited test capabilites
allowed me to conclude that there's problem with 2.4.20
kernel. Everything works fine with 2.6.0 with USAGI patches. Patched 
kernels with enhanced IPv6 support can be downloaded from
\url{http://www.linux-ipv6.org/}. Please let me know if your kernel
works or not.

\subsection{Windows specific}

\Q After installing \emph{Advanced Networking Pack} or \emph{Windows XP
  ServicePack2} my DHCPv6 (or other IPv6 application) stopped
   working. Is Dibbler XP SP2 compatible?

\A In both of this products, there is a IPv6 firewall installed. It
is configured by default to reject all incoming IPv6 traffic. You have
to disable this firewall. To do so, issue following commands in a
console:

\begin{verbatim}

\end{verbatim}

