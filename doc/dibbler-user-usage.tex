%%
%% $Id: dibbler-user-usage.tex,v 1.3 2004-06-19 19:51:14 thomson Exp $
%%
%% $Log: not supported by cvs2svn $
%% Revision 1.2  2004/06/19 10:24:59  thomson
%% Hyperlinks in PDF, building process modified
%%

\section{Installation and usage}
Both client and server are installed in the same way. Installation
method is different in WindowsXP and Linux systems, so they're described
separately. To simplify installation, it assumes that binary versions
are used\footnote{Compilation is not
  required, binary version can be used safely. Compilation can be performed by
  advanced users, see \emph{Compilation} section for details.}.

\subsection{Linux installation}
Obtain (e.g. download from \url{http://klub.com.pl/dhcpv6/}) an archive with
Dibbler binaries and extract it to \verb+/var/lib/dibbler+ directory:
\begin{verbatim}
cd /var/lib/dibbler/
tar zxvf dibbler-0.2.0-linux.tar.gz 
\end{verbatim}

Depending what functionality do you want to use (server or client),
you should edit config file (\verb+client.conf+ for client and \verb+server.conf+
for server). After editing, issue following commands:

\begin{verbatim}
./dibbler-server-linux start
\end{verbatim}

for starting server or

\begin{verbatim}
./dibbler-client-linux start
\end{verbatim}

for starting client. \verb+start+ parameter needs a little comment. It
instructs Dibbler to run in daemon mode -- detach from console and run
in the background. During config files fine-tuning, it is ofter better
to watch Dibbler's bahavior instantly. In this case, use \verb+run+
instead of \verb+start+ parameter. Dibbler will present its messages on
your console. To finish it, press ctrl-c.

To stop server running in daemon mode, type:
\begin{verbatim}
./dibbler-server-linux stop
\end{verbatim}

To stop client running in daemon mode, type:
\begin{verbatim}
./dibbler-client-linux stop
\end{verbatim}

To see, if client or server are running\footnote{Running status is
  based on /var/lib/dibbler/client.pid or server.pid files. In rare
  occasions, when server crashes, this status will show server status as running.}, type:
\begin{verbatim}
./dibbler-client-linux status
\end{verbatim}

or
\begin{verbatim}
./dibbler-server-linux status
\end{verbatim}

\subsection{WindowsXP installation}
Obtain (e.g. download from \url{http://klub.com.pl/dhcpv6/}) an archive with
Dibbler binaries and extract it to a directory, e.g. \verb+c:\dibbler+.

Open a console (start $\rightarrow$ run... $\rightarrow$ cmd) and
issue following commands:

\begin{verbatim}
c:
cd \dibbler
dibbler-client-winxp run -d c:\dibbler
\end{verbatim}

\verb+run+ command instructs Dibbler to run in console and not become a
daemon. Is it very useful feature in early configuration stages. To
finish it, press ctrl-c. 

After configuration files are edited and tested, user can install Dibbler as a
Windows service. In this case, following command should be issued:
\begin{verbatim}
dibbler-client-winxp install -d c:\dibbler
\end{verbatim}

Now dibbler is installed as a Windows service and it can be controlled
from Control Panel $\rightarrow$ Administrative tasks $\rightarrow$
Services. 

If you want to uninstall Dibbler as a service, use \verb+uninstall+
instead of \verb+install+.

\section{Compilation}
Dibbler is distributed in 2 versions: binary and source files. For
most users, binary version is better choice.  Compilation is
performed by more experienced users, preferably with programming
knowledge. It does not offer any advances over binary version, only
allows to understand internal Dibbler workings. You probalby want just
install and use Dibbler. If that is your case, read section
named \emph{Installation}.

\subsection{Linux compilation}
Issue following commands:
\begin{verbatim}
tar zxvf dibbler-0.2.0-src.tar.gz
cd dibber
make
\end{verbatim}
That's it. If there are problems with missing/different compiler
version, take a look at Makefile.inc file. Dibbler was compiled using
gcc 2.95, 3.0, 3.2, 3.3 and 3.4 versions. Lexer files were generated using
flex 2.5.31. Parser file were created using bison++
1.21.9\footnote{flex and bison++ tools are not required to compile
  Dibbler. Generated files are placed in CVS and in tar.gz
  archives}. Everything was developed under Debian GNU/Linux systems.

\subsection{WindowsXP compilation}
Download dibbler-0.2.0-src.tar.gz and extract it. In port-winxp there
will be project files (one for server and one for client) for MS
Visual C++ 2003. Open one of them and click Build command. That should
do the trick.

\subsubsection{Flex/bison under Windows}

Lexer and Parser files (\verb+ClntLexer.*+, \verb+ClntParser.*+, \verb+SrvLexer.*+ and
\verb+SrvParser.*+) are generated by author and placed in CVS and
archives. There is no need to generate them. However, if you insist on
doing so, there is an flex and bison binary included in port-winxp. Take note that
several modifications are required:

\begin{itemize}
\item To generate \verb+ClntParser.cpp+ and \verb+ClntLexer.cpp+ files, you can use
\verb+parser.bat+. After generation, in file \verb+ClntLexer.cpp+ replace: \verb+class istream;+
with: \verb+#include <iostream>+ and \verb+using namespace std;+ lines.
\item flex binary included is slightly modified. It generates
\verb+#include "FlexLexer.h"+ instead of \verb+#include <FlexLexer.h>+. You should
add .\ to include path if you have problem with missing \verb+FlexLexer.h+.
Also note that \verb+FlexLexer.h+ is modified (std:: added in several places,
\verb+<fstream.h>+ is replaced with \verb+<fstream>+ etc.)
%%\item In file ClntParser.cpp, substitute line (around 1860): ,,	*++yyvsp = yylval;''
%%with: ,,*++yyvsp = ::yylval;''. This trick is supposed to fix numerous
%%parser problems.
\end{itemize}

Keep in mind that author is in no way a flex/bison guru and found this method
in a painful trial-and-error way. 
