%%
%% Dibbler - a portable DHCPv6
%%
%% authors: Tomasz Mrugalski <thomson@klub.com.pl>
%%
%%
%% released under GNU GPL v2 licence
%%
%% $Id: dibbler-user-usage.tex,v 1.18 2006-10-04 22:03:35 thomson Exp $
%%

\section{Installation and usage}
Client, server and relay are installed in the same way. Installation
method is different in Windows and Linux systems, so each system
installation is described separately. To simplify installation, it
assumes that binary versions are used\footnote{Compilation is not
required, usually binary version can be used. Compilation should be
performed by advanced users only, see \emph{Compilation} section for
details.}.

\subsection{Linux installation}
Starting with 0.4.0, Dibbler consists of 3 different elements: client,
server and relay. During writing this documentation, Dibbler is already
present in following Linux distributions:
\begin{description}
 \item[\href{http://debian.org}{Debian GNU/Linux}] -- use standard tools
 (apt-get, aptitude) to install dibbler-client, dibbler-server,
 dibbler-relay or dibbler-doc packages (e.g. apt-get install dibbler-client)
 \item[\href{http://www.gentoo.org}{Gentoo Linux}] -- use emerge to
 install dibbler (e.g. emerge dibbler).
 \item[\href{http://www.pld-linux.org}{PLD GNU/Linux}] -- use standard
  PLD's poldek tool to install dibbler package.
\end{description}

If you are using other Linux distribution, obtain (e.g. download from
\url{http://klub.com.pl/dhcpv6/}) an archive, which suits your
needs. Currently there are available RPM packages
(which can be used in RedHat, Fedora Core, Mandrake or PLD
distribution), DEB packages (suitable for Debian, Ubuntu or Knoppix) and
ebuild (for Gentoo users). To install rpm package, execute 
\verb+rpm -i archive.rpm+ command. For example, to install dibbler 0.4.1,
issue following command:

\begin{verbatim}
rpm -i dibbler-0.4.1-1.i386.rpm 
\end{verbatim}

To install Dibbler on Debian or other system with dpkg management
system, run $dpkg -i archive.deb$ command. For example, to install
server, issue following command:

\begin{verbatim}
dpkg -i dibbler-server_0.4.1-1_i386.deb
\end{verbatim}

To install Dibbler in Gentoo systems, just type:

\begin{verbatim}
emerge dibbler
\end{verbatim}

If you would like to install Dibbler from sources, please download tar.gz
source archive, extract it, type make followed by target (e.g. server,
client or relay\footnote{To get full target list, type: make help.}). 
After successful compilation type make install. For example, to build
server and relay, type:

\begin{verbatim}
tar zxvf dibbler-0.4.0-src.tar.gz
make server relay
make install
mkdir -p /var/lib/dibbler
\end{verbatim}

Depending what functionality do you want to use (server,client or relay),
you should edit configuration file (\verb+client.conf+ for client, \verb+server.conf+
for server and \verb+relay.conf+ for relay). All configuration files should
be placed in the \verb+/etc/dibbler+ directory. Also make sure that 
\verb+/var/lib/dibbler+ directory is present and is writeable. After
editing configuration files, issue one of the following commands:

\begin{verbatim}
dibbler-server start
dibbler-client start
dibbler-relay start
\end{verbatim}

\verb+start+ parameter requires little explanation. It
instructs Dibbler to run in daemon mode -- detach from console and run
in the background. During configuration files fine-tuning, it is ofter better
to watch Dibbler's bahavior instantly. In this case, use \verb+run+
instead of \verb+start+ parameter. Dibbler will present its messages on
your console instead of log files. To finish it, press ctrl-c.

To stop server, client or relay running in daemon mode, type:
\begin{verbatim}
dibbler-server stop
dibbler-client stop
dibbler-relay stop
\end{verbatim}

To see, if client, server or relay are running, type:

\begin{verbatim}
dibbler-server status
dibbler-client status
dibbler-relay status
\end{verbatim}

To see full list of available commands, type \verb+dibbler-server+, 
\verb+dibbler-client+ or \verb+dibbler-relay+ without any parameters.

\subsection{Windows installation}
Since the 0.2.1-RC1 release, Dibbler supports Windows XP and 2003. 
Support for Vista was added somewhere around 0.7.x. Support for
Windows 7 was added in 0.8.0RC1. In version 0.4.1 exprimental support for Windows NT4 and 2000 was added. The easiest
way of Windows installation is to download clickable Windows installer. It can be downloaded from
\url{http://klub.com.pl/dhcpv6/}. After downloading, click on it and
follow on screen instructions. Dibbler will be installed and all
required links will be placed in the Start menu. Note that there are
two Windows versions (ports): one for modern systems (XP/2003/Vista
and Win7) and one for archaic ones (NT4/2000). Make sure
to use proper port. If you haven't set up IPv6 support, see following
sections for details.

\subsection{IPv6 support}
Some systems does not have IPv6 enabled by default. In that is the case,
you can skip following subsections safely. If you are not sure, here is
an easy way to check it. To verify if you have IPv6 support, execute
following command: \verb+ping6 ::1+ (Linux) or \verb+ping ::1+
(Windows). If you get replies, you have IPv6 already installed.

\subsubsection{Setting up IPv6 in Linux}
IPv6 can be enabled in Linux systems in two ways: compiled directly
into kernel or as a module. If you don't have IPv6 enabled, try to load IPv6 module:
\verb+modprobe ipv6+ (command executed as root) and try ping6 once more. If that
fails, you have to recompile kernel to support IPv6. There are 
numerous descriptions how to recompile kernel available on the web, just
type "kernel compilation howto" in \href{http://www.google.com}{Google}.

\subsubsection{Setting up IPv6 in Windows Vista and Win7}
Both systems have IPv6 enabled by default. Also note that Win7 also
has DHCPv6 client built-in, so you may use it as well.

\subsubsection{Setting up IPv6 in WindowsXP and 2003}
If you have already working IPv6 support, you can safely skip this section.
The easiest way to enable IPv6 support is to right click on the
\verb+My network place+ on the desktop, select \verb+Properties+, then locate
your network interface, right click it and select \verb+Properties+. Then
click \verb+Install...+, choose protocol and then IPv6 (its naming is
somewhat diffrent depending on what Service Pack you have installed).
In XP, there's much quicker way to install IPv6. Simply run command
\verb+ipv6 install+ (i.e. hit Start..., choose run... and then type 
\verb+ipv6 install+). Also make sure that you have built-in firewall
disabled. See \emph{Frequently Asked Question} section for details.

\subsubsection{Setting up IPv6 in Windows 2000}
If you have already working IPv6 support, you can safely skip this
section. The following description was provided by Sob (
(\href{mailto:sob(at)hisoftware.cz}{sob(at)hisoftware.cz}). Thanks. This
description assumes that ServicePack 4 is already installed.

\begin{enumerate}
  \item Download the file tpipv6-001205.exe from:
    \url{http://msdn.microsoft.com/downloads/sdks/platform/tpipv6.asp}
    and save it to a local folder (for example, \verb+C:\IPv6TP+).
  \item From the local folder (\verb+C:\IPv6TP+), run \verb+Tpipv6-001205.exe+ and extract the
    files to the same location.
  \item From the local folder (\verb+C:\IPv6TP+), run \verb+Setup.exe -x+ and extract the files to
    a subfolder of the current folder (for example, \verb+C:\IPv6TP\files+).
  \item From the folder containing the extracted files (\verb+C:\IPv6TP\files+), open the
    file \verb+Hotfix.inf+ in a text editor.
  \item In the [Version] section of the Hotfix.inf file, change the line
    NTServicePackVersion=256 to NTServicePackVersion=1024, and then
    save changes. \footnote{This defines Service Pack requirement.
      NTServicePackVersion is a ServicePack version multiplied by 256. If there
    would be SP5 available, this value should have been changed to the 1280.}
  \item From the folder containing the extracted files (\verb+C:\IPv6TP\files+), run
    \verb+Hotfix.exe+.
  \item Restart the computer when prompted.
  \item After the computer is restarted, from the Windows 2000 desktop, click Start,
    point to Settings, and then click Network and Dial-up Connections. As an
    alternative, you can right-click My Network Places, and then click Properties.
  \item Right-click the Ethernet-based network interface to which you want to add the IPv6
    protocol, and then click Properties. Typically, this network interface is named
    Local Area Connection.
  \item Click Install.
  \item In the Select Network Component Type dialog box, click Protocol, and then
    click Add.
  \item In the Select Network Protocol dialog box, click Microsoft IPv6 Protocol and
    then click OK.
  \item Click Close to close the Local Area Connection Properties
    dialog box.
\end{enumerate}

\subsubsection{Setting up IPv6 in Windows NT4}
If you have already working IPv6 support, you can safely skip this section.
The following description was provided by The following description was provided by Sob
(\href{mailto:sob(at)hisoftware.cz}{sob(at)hisoftware.cz}). Thanks.

\begin{enumerate}
  \item Download the file msripv6-bin-1-4.exe from:
    \url{http://research.microsoft.com/msripv6/msripv6.htm}{Microsoft}
    and save it to a local folder (for example, \verb+C:\IPv6Kit+).
  \item From the local folder (\verb+C:\IPv6Kit+), run \verb+msripv6-bin-1-4.exe+ and extract the
    files to the same location.
  \item Start the Control Panel's "Network" applet (an alternative way to do this is
    to right-click on "Network Neighborhood" and select "Properties") and select
    the "Protocols" tab.
  \item Click the "Add..." button and then "Have Disk...". When it asks you for
    a disk, give it the full pathname to where you downloaded the binary
    distribution kit (\verb+C:\IPv6Kit+).
  \item IPv6 is now installed.
\end{enumerate}

\subsection{Compilation}
Dibbler is distributed in 2 versions: binary and source code. For
most users, binary version is better choice.  Compilation is
performed by more experienced users, preferably with programming
knowledge. It does not offer significant advantages over binary version,
only allows to understand internal Dibbler workings. You probably want
just install and use Dibbler. If that is your case, read section
named \emph{Installation}. However, if you are skilled enough, you might
want to tune several Dibbler aspects during compilation. See \emph{
Dibbler Developer's Guide} for information about various compilation parameters.

\subsubsection{Linux compilation}

Compilation in most cases is not necessary and should be performed
only by experienced users. To compile dibbler, issue following commands:
\begin{verbatim}
tar zxvf dibbler-0.4.0-src.tar.gz
cd dibbler
make server client relay doc
\end{verbatim}
That's it. You can also install it in the system by issuing command:

\begin{verbatim}
make install
\end{verbatim}

If there are problems with missing/different compiler
version, take a look at the beginning of the Makefile.inc
file. Dibbler was compiled using gcc 2.95, 3.0, 3.2, 3.3, 3.4, 4.0 and 4.1
versions. Note that 2.95 is now considered obsolete and was not tested
for some time. Lexer files were generated using flex 2.5.33. Parser file
were created using bison++ 1.21.9\footnote{flex and bison++ tools are
  not required to compile Dibbler. Generated files are placed in CVS
  and in tar.gz archives}. 

If there are problems with \verb+SrvLexer.cpp+ and
\verb+ClntLexer.cpp+ files, please use FlexLexer.h in Port-linux/
directory. Most simple way to do this is to copy this file to
\verb+/usr/include+ directory. 

\subsubsection{Modern Windows (XP...Vista) compilation}
Download dibbler-\version-src.tar.gz and extract it. In \verb+Port-win32+ there
are several project files (for server, client and relay) for MS
Visual Studio 2003. Previous dibbler releases were compiled using MS
Visual Studio .NET (sometimes called 2002). It might work with newest
dibber version, but there are no guarantee. Open one of the project
files and click Build command. That should start compilation. After a
while, binary exe files will be stored in the \verb+Debug/+ directory.

\subsubsection{Legacy Windows (NT/2000) compilation}
Windows NT4/2000 port is considered experimental, but there are reports
that it works just fine. To compile it, you should download dev-cpp
(\url{http://www.bloodshed.net/dev/devcpp.html}), a free IDE for
Windows utilising minGW port of the gcc for Windows. Run dev-cpp,
click ,,open project...'', and open one of the \verb+*.dev+ files located
in the Port-winnt2k directory, then click compile. You also should
take a look at \verb+Port-winnt2k/INFO+ file for details.

