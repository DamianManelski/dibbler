%%
%% $Id: dibbler-user-usage.tex,v 1.8 2005-01-23 23:16:56 thomson Exp $
%%
%% $Log: not supported by cvs2svn $
%% Revision 1.7  2004/12/27 20:44:29  thomson
%% Date is now generated in an automatic manner, deb/rpm building described
%%
%% Revision 1.6  2004/12/08 00:20:57  thomson
%% Binary file names information changed.
%%
%% Revision 1.5  2004/10/25 20:45:54  thomson
%% Option support, parsers rewritten. ClntIfaceMgr now handles options.
%%
%% Revision 1.4  2004/07/05 00:12:30  thomson
%% Lots of minor changes.
%%
%% Revision 1.3  2004/06/19 19:51:14  thomson
%% Various fixes.
%%
%% Revision 1.2  2004/06/19 10:24:59  thomson
%% Hyperlinks in PDF, building process modified
%%

\section{Installation and usage}
Both client and server are installed in the same way. Installation
method is different in WindowsXP and Linux systems, so they're described
separately. To simplify installation, it assumes that binary versions
are used\footnote{Compilation is not
  required, binary version can be used safely. Compilation can be performed by
  advanced users, see \emph{Compilation} section for details.}.

\subsection{Linux installation}
As for 0.3.1, both server and client were provided within one
packege. Starting with 0.4.0, there will be 3 different packages:
client, server and relay. For some architectures there will be also
documentation package provided.

Obtain (e.g. download from \url{http://klub.com.pl/dhcpv6/}) an
archive, which suits your needs. Currently there are provided RPM packages
(which can be used in RedHat, Fedora Core, Mandrake or PLD
distribution), DEB packages (suitable for Debian or Knoppix) and
ebuild (for Gentoo users). To install rpm package, run $rpm -i
archive.rpm$ command. For example, to install dibbler 0.3.1, issue following command:
\begin{verbatim}
rpm -i dibbler-0.3.1-1.i386.rpm 
\end{verbatim}

To install Dibbler on Debian or other system with dpkg management
system, run $dpkg -i archive.deb$ command. For example, to install
dibbler 0.4.0, issue following command:

\begin{verbatim}
dpkg -i dibbler_0.4.0_i386.rpm 
\end{verbatim}

To install Dibbler in Gentoo systems, download ebuild script and issue
command:

\begin{verbatim}
emerge dibbler-0.3.1.ebuild
\end{verbatim}

If you would like to install Dibbler from sources, download tar.gz
source archive, extract it, type make followed by target (e.g. server,
client or relay\footnote{To get full target list, type: $make
  help$}). After successful compilation type make install. For
example, to build server and relay, type:

\begin{verbatim}
tar zxvf dibbler-0.3.1-src.tar.gz
make server relay
make install
\end{verbatim}

Depending what functionality do you want to use (server,client or relay),
you should edit config file (\verb+client.conf+ for client, \verb+server.conf+
for server and \verb+relay.conf+ for relay). All config files should
be placed in the +/etc/dibbler+ directory. After editing, issue one of
the following commands:

\begin{verbatim}
dibbler-server start
dibbler-client start
dibbler-relay start
\end{verbatim}

\verb+start+ parameter needs a little comment. It
instructs Dibbler to run in daemon mode -- detach from console and run
in the background. During config files fine-tuning, it is ofter better
to watch Dibbler's bahavior instantly. In this case, use \verb+run+
instead of \verb+start+ parameter. Dibbler will present its messages on
your console. To finish it, press ctrl-c.

To stop server, client or relay running in daemon mode, type:
\begin{verbatim}
dibbler-server stop
dibbler-client stop
dibbler-relay stop
\end{verbatim}

To see, if client or server are running\footnote{Running status is
  based on /var/lib/dibbler/client.pid or server.pid files. In rare
  occasions, when server crashes, this status will show server status as running.}, type:
\begin{verbatim}
dibbler-server status
dibbler-client status
dibbler-relay status
\end{verbatim}

\subsection{Windows installation}
Starting at 0.2.1, Dibbler supports Windows XP and 2003. 
Obtain (e.g. download from \url{http://klub.com.pl/dhcpv6/}) an
windows installer. After downloading, click on it and follow on screen
instructions. Dibbler will be installed and all required links will be
placed in the Start menu.

\section{Compilation}
Dibbler is distributed in 2 versions: binary and source files. For
most users, binary version is better choice.  Compilation is
performed by more experienced users, preferably with programming
knowledge. It does not offer any advances over binary version, only
allows to understand internal Dibbler workings. You probalby want just
install and use Dibbler. If that is your case, read section
named \emph{Installation}.

\subsection{Linux compilation}

Compilation in most cases is not necessary and should be performed
only by experienced users. Perferred method is to use binaries
provided on Dibbler's website. Issue following commands:
\begin{verbatim}
tar zxvf dibbler-0.3.1-src.tar.gz
cd dibber
make server client
\end{verbatim}
That's it. You can also install it in the system by issuing command:

\begin{verbatim}
make install
\end{verbatim}

If there are problems with missing/different compiler
version, take a look at the beginning of the Makefile.inc
file. Dibbler was compiled using gcc 2.95, 3.0, 3.2, 3.3 and 3.4
versions. Lexer files were generated using flex 2.5.31. Parser file
were created using bison++ 1.21.9\footnote{flex and bison++ tools are
  not required to compile Dibbler. Generated files are placed in CVS
  and in tar.gz archives}. Everything was developed under Debian
GNU/Linux system.

If there are problems with \verb+SrvLexer.cpp+ and
\verb+ClntLexer.cpp+ files, please use FlexLexer.h in Port-linux/
directory. Most simple way to do this is to copy this file to
\verb+/usr/include+ directory. Additional information about
compilation can be found in \emph{Dibbler Developer's Guide}.

\subsection{WindowsXP compilation}
Download dibbler-0.3.0-src.tar.gz and extract it. In Port-winxp there
will be project files (for server, client and relay) for MS
Visual C++ 2003. Open one of them and click Build command. That should
do the trick. Additional information about compilation can be found in
\emph{Dibbler Developer's Guide}.
