%%
%% Dibbler - a portable DHCPv6
%%
%% authors: Tomasz Mrugalski <thomson@klub.com.pl>
%%          Marek Senderski <msend@o2.pl>
%%
%% released under GNU GPL v2 or later licence
%%
%% $Id: dibbler-devel.tex,v 1.8 2004-12-01 20:55:49 thomson Exp $
%%
%% $Log: not supported by cvs2svn $
%% Revision 1.7  2004/11/30 00:52:05  thomson
%% Minor changes.
%%
%% Revision 1.6  2004/11/25 01:16:36  thomson
%% *** empty log message ***
%%
%% Revision 1.5  2004/11/15 21:11:18  thomson
%% *** empty log message ***
%%
%% Revision 1.4  2004/11/02 00:02:22  thomson
%% Documentation improved.
%%
%% Revision 1.3  2004/10/25 20:45:54  thomson
%% Option support, parsers rewritten. ClntIfaceMgr now handles options.
%%
%% Revision 1.2  2004/07/05 01:04:40  thomson
%% Initial version.
%%
%%

\documentclass[10pt]{article}
\usepackage[latin2]{inputenc}
\usepackage[dvips]{graphicx}
\usepackage{float}
\usepackage{makeidx}
\usepackage{fancyhdr}
\usepackage{indentfirst}
\usepackage{longtable}
\usepackage{url}
\usepackage[usenames]{color}
\usepackage[ps2pdf,colorlinks=true,linkcolor=myLinkColor,urlcolor=myUrlColor]{hyperref}

\definecolor{myBgColor}{rgb}{0.9,0.9,0.9}
\definecolor{myLinkColor}{rgb}{0.0,0.4,0.0}
\definecolor{myUrlColor}{rgb}{0.7,0.2, 0.4}
\pagecolor{myBgColor}

%\setlength{\topmargin}{-1.0cm}

\addtolength{\textwidth}{4cm}
\addtolength{\hoffset}{-2cm} 
\addtolength{\textheight}{2.0 cm}
\addtolength{\voffset}{-0.5cm}

\author{Tomasz Mrugalski\\ \small{\href{mailto:thomson(at)klub.com.pl}{thomson(at)klub.com.pl}}}
\date{2004-11-15}
\title{Dibbler -- a portable DHCPv6\\Developer's Guide}

\pagestyle{fancy}
\fancyhf{}
\fancyhead[L]{\small\bfseries Dibbler -- a portable DHCPv6}
\fancyhead[C]{\small\bfseries Developer's Guide}
\fancyhead[R]{\small\bfseries\thepage}
%%\fancyhead[C]{\small\bfseries\leftmark}
\renewcommand{\headrulewidth}{0.5pt}
\renewcommand{\footrulewidth}{0pt}
\addtolength{\headheight}{0.5pt}
\fancypagestyle{plain}{%
\fancyhead{} %
\renewcommand{\headrulewidth}{0pt} %
}


\makeindex

\newcommand{\msg}[1]{ \textbf{#1} }
\newcommand{\opt}[1]{ \emph{#1} }
\newcommand{\IA}{ \textbf{\emph{IA}} }
\newcommand{\duid} { DUID }
\newcommand{\code}[1]{\textbf{#1}}
\newcommand{\subsubsubsection}[1]{\subsubsection{#1}}
%% pakiety wypisuj tak: \msgt{SOLICIT}
%% opcje wypisuj tak:   \opt{Option Request}

\begin{document}

\vspace{-2cm}
\maketitle
\vspace{-1cm}
\begin{center}
version 0.3.0-RC3
\end{center}

\tableofcontents

This document is currently being written. First full version will be
included in 0.3.0 release.

%% INTRO, OVERVIEW
%%
%% Dibbler - a portable DHCPv6
%%
%% authors: Tomasz Mrugalski <thomson@klub.com.pl>
%%          Marek Senderski <msend@o2.pl>
%%
%% released under GNU GPL v2 or later licence
%%
%% $Id: dibbler-devel-intro.tex,v 1.6 2006-07-16 11:24:51 thomson Exp $
%%
%% $Log: not supported by cvs2svn $
%% Revision 1.5  2004/12/01 20:55:49  thomson
%% Documentation updated.
%%
%% Revision 1.4  2004/11/30 00:52:05  thomson
%% Minor changes.
%%
%% Revision 1.3  2004/11/28 11:14:07  thomson
%% RFCs and drafts added, clarification about option values
%%
%% Revision 1.2  2004/11/25 01:16:36  thomson
%% *** empty log message ***
%%
%% Revision 1.1  2004/07/05 01:04:40  thomson
%% Initial version.
%%
%%

\section{Intro}
Welcome to the Dibbler developer's guide. This document describes
various aspects of the compilation and installation of Dibbler server
and client. Detailed description of the internal architecture is also
provided. People with programming background can find useful
informations here. Main purpose of this document is to help
contributors to quickly know Dibbler from the inside.

This document is intenteded just as its title states -- a guide. It is
not a thorough code description. To quickly wander around classes and
methods used, see documentation generated with the Doxygen tool (open
file \verb+doc/html/index.html+). More informations about
documentation is provided in section \ref{doc}.


%% COMPILATION
%%
%% Dibbler - a portable DHCPv6
%%
%% authors: Tomasz Mrugalski <thomson@klub.com.pl>
%%          Marek Senderski <msend@o2.pl>
%%
%% released under GNU GPL v2 or later licence
%%
%% $Id: dibbler-devel-compile.tex,v 1.4 2004-11-25 01:16:36 thomson Exp $
%%
%% $Log: not supported by cvs2svn $
%% Revision 1.3  2004/10/25 20:45:54  thomson
%% Option support, parsers rewritten. ClntIfaceMgr now handles options.
%%
%% Revision 1.2  2004/09/28 16:02:09  thomson
%% no message
%%
%% Revision 1.1  2004/07/05 01:04:40  thomson
%% Initial version.
%%
%%

\section{Compilation}
Currently Dibbler supports two platforms: Linux with kernels 2.4 and
2.6 series and Windows (XP and 2003). Compilation process is system
dependent, so it is described for Linux and Windows separately.

\subsection{Linux}
To compile Dibbler, extract sources, and type:
\begin{verbatim}
make client
make server
\end{verbatim}
to build client and server. Although parser files are generated using
flex and bison++, those generated sources are included and there is
no need to generate them. To generate If someone wants to generate it by hand
instead of using those supplied versions, here are appropriate commands:
\begin{verbatim}
cd ClntCfgMgr
make parser
\end{verbatim}
to generate client parser, and:
\begin{verbatim}
cd SrvCfgMgr
make parser
\end{verbatim}
to generate server parser.

There occassionaly might be problem with compilation, when different
flex version is installed in the system. Proper FlexLexer.h is
provided in the SrvCfgMgr and ClntCfgMgr directories.

\subsubsection{Documentation}
To generate documentation in \LaTeX, issue command: 
\begin{verbatim}
make doc
\end{verbatim}

There is also documentation generated from the sources
itseft. It is being done with the
\href{http://www.doxygen.org}{Doxygen} tool. Make sure that you
have doxygen installed. To generate this documentation, issue command:
\begin{verbatim}
make oxygen
\end{verbatim}

\subsection{Windows}
To compile Dibbler under Windows, MS Visual Studio 2003 was
used. Project files are provided in the CVS and source archives.

Select project name (server-winxp or client-winxp), click properties,
choose ,,Debugging'' from ,,Configuration Properties''. Adjust ,,Command
arguments'' to meet your directory.

If you are using MS Visual Studio 2003, there might be a problem with
lowlevel-win32.c file compilation. Compiler might complain about
missing Ipv6IfIndex in \_IP\_ADDAPTER\_ADDRESSES. There is a simple way
to bypass this.  In \verb+Program Files/Microsoft Visual Studio .NET/Vc7/PlatformSDK/Include/+ directory, there is \verb+IPTypes.h+
file. It contains structure:

\begin{verbatim}
typedef struct _IP_ADAPTER_ADDRESSES {
    union {
        ULONGLONG Alignment;
        struct {
            ULONG Length;
            DWORD IfIndex;
        };
    };
    struct _IP_ADAPTER_ADDRESSES *Next;
    PCHAR AdapterName;
    PIP_ADAPTER_UNICAST_ADDRESS FirstUnicastAddress;
    PIP_ADAPTER_ANYCAST_ADDRESS FirstAnycastAddress;
    PIP_ADAPTER_MULTICAST_ADDRESS FirstMulticastAddress;
    PIP_ADAPTER_DNS_SERVER_ADDRESS FirstDnsServerAddress;
    PWCHAR DnsSuffix;
    PWCHAR Description;
    PWCHAR FriendlyName;
    BYTE PhysicalAddress[MAX_ADAPTER_ADDRESS_LENGTH];
    DWORD PhysicalAddressLength;
    DWORD Flags;
    DWORD Mtu;
    DWORD IfType;
    IF_OPER_STATUS OperStatus;
} IP_ADAPTER_ADDRESSES, *PIP_ADAPTER_ADDRESSES;
\end{verbatim}

You should slightly modify it. Just add on additional field:
\verb+DWORD Ipv6IfIndex;+. Now it should look like this:

\begin{verbatim}
typedef struct _IP_ADAPTER_ADDRESSES {
    union {
        ULONGLONG Alignment;
        struct {
            ULONG Length;
            DWORD IfIndex;
        };
    };
    struct _IP_ADAPTER_ADDRESSES *Next;
    PCHAR AdapterName;
    PIP_ADAPTER_UNICAST_ADDRESS FirstUnicastAddress;
    PIP_ADAPTER_ANYCAST_ADDRESS FirstAnycastAddress;
    PIP_ADAPTER_MULTICAST_ADDRESS FirstMulticastAddress;
    PIP_ADAPTER_DNS_SERVER_ADDRESS FirstDnsServerAddress;
    PWCHAR DnsSuffix;
    PWCHAR Description;
    PWCHAR FriendlyName;
    BYTE PhysicalAddress[MAX_ADAPTER_ADDRESS_LENGTH];
    DWORD PhysicalAddressLength;
    DWORD Flags;
    DWORD Mtu;
    DWORD IfType;
    IF_OPER_STATUS OperStatus;
    DWORD Ipv6IfIndex;
} IP_ADAPTER_ADDRESSES, *PIP_ADAPTER_ADDRESSES;
\end{verbatim}

\subsubsection{Flex/bison under Windows}
As was pointed out before, flex and bison++ tools are not required to
successfully build Dibbler. They are only required, if changes are
made to the parsers. Lexer and Parser files (\verb+ClntLexer.*+, \verb+ClntParser.*+, \verb+SrvLexer.*+ and
\verb+SrvParser.*+) are generated by author and placed in CVS and
archives. There is no need to generate them. However, if you insist on
doing so, there is an flex and bison binary included in port-winxp. Take note that
several modifications are required:

\begin{itemize}
\item To generate \verb+ClntParser.cpp+ and \verb+ClntLexer.cpp+ files, you can use
\verb+parser.bat+. After generation, in file \verb+ClntLexer.cpp+ replace: \verb+class istream;+
with: \verb+#include <iostream>+ and \verb+using namespace std;+ lines.
\item flex binary included is slightly modified. It generates

\begin{verbatim}
#include "FlexLexer.h"
\end{verbatim} 
instead of 
\begin{verbatim}
#include <FlexLexer.h>
\end{verbatim} 

You should
add .\ to include path if you have problem with missing \verb+FlexLexer.h+.
Also note that \verb+FlexLexer.h+ is modified (std:: added in several places,
\verb+<fstream.h>+ is replaced with \verb+<fstream>+ etc.)
%%\item In file ClntParser.cpp, substitute line (around 1860): ,,	*++yyvsp = yylval;''
%%with: ,,*++yyvsp = ::yylval;''. This trick is supposed to fix numerous
%%parser problems.
\end{itemize}

Keep in mind that author is in no way a flex/bison guru and found this method
in a painful trial-and-error way. 


%% COMMON INFORMATION (e.g. SmartPtr)
%%
%% Dibbler - a portable DHCPv6
%%
%% authors: Tomasz Mrugalski <thomson@klub.com.pl>
%%          Marek Senderski <msend@o2.pl>
%%
%% released under GNU GPL v2 or later licence
%%
%%
%% $Id: dibbler-devel-common.tex,v 1.9 2006-07-31 02:10:07 thomson Exp $
%%

\section{General information}
This section covers several loosely related topics.

\subsection{Release cycle}
Dibbler is being released as a one product, i.e. client, server and
relay are always released together. Each version is being designated
with three numbers, separated by periods, e.g. 0.4.2. Every time a new
significant functionality is added, the middle number is being
increased. When new release contains only fixes and small
improvements, only the minor number is changed. Leftmost number is
currently set to 0 as not all features mentioned in base DHCPv6
document (RFC3315, \cite{rfc3315}) are implemented. When this
implementation will be complete, release number will reach
1.0.0. Since DHCPv6 specification is extensive, don't expect this to
happen anytime soon.

\subsection{Documentation}
\label{doc}
There are three parts of the documentation: User's Guide, Developer's
Guide and a Code documentation. Both guides are written in \LaTeX
(*.tex files). To generate PDF files, you need to have \LaTeX
installed. To generate Code documentation, 
a tool called \href{http://www.doxygen.org}{Doxygen} is required. All
documentation is of course available at
\href{http://klub.com.pl/dhcpv6}{Dibbler's homepage}.

To generate all documentation, type (in Dibbler source directory):
\begin{verbatim}
make doc oxygen
\end{verbatim}

In this section various common aspects of the Dibbler internal
workings are decribed.

\subsection{Memory/CPU usage}
This section provides basic insight about memory and CPU requirements
for the dibbler components. 

Folowing paragraphs describe memory and CPU usage measurements. They
were taken on a AMD Athlon 2800+ (actual clock speed: 2083MHz),
running under Linux 2.6.17.3. Dibbler was compiled by gcc 4.1.2
(exact version number printed by \verb+gcc --version+ command: \\
\verb+gcc (GCC) 4.1.2 20060715 (prerelease) (Debian 4.1.1-9)+).

Every Dibbler component (client, server or relay) is event driven. It
means that it does nothing unless some data was received or a specific
timeout has been reached. Each component most of the time spends in a
\verb+select()+ system call. This means that (unless lots of traffic
is being received) actual CPU usage is 0. During tests, author was
unable to observe any CPU consumption above 0,0\%. 

In the 0.5.0 release, a compilation options called Modular features
was added (see section \ref{modular-features}). One of the possible
way of compiling Dibbler is to disable poslib - a library used to
perform DNS Updates. Dibbler binaries compiled without poslib are
designated as -wo-poslib. It is possible to compile Dibbler with
various compilation options. In particular (enabled by default)
\verb+-g+ option includes debugging information in the binary file
(this greatly affects binary file size, but does not affect memory
usage), -O0 (disably any kind of optimisation) or -Os (produce
smallest possible code). Debugging informations can be removed using
\verb+strip+ command (designated below as -stripped). Linux command
line tool called \verb+top+ was used to measure memory usage. VIRT is
a virtual memory size, RES denotes size of actual physical memory used
and SHR is a size of a shared memory. See top manual page for details.

\begin{center}
\begin{tabular}{|c|c|c|c|c|c|l|l|}
\hline
VIRT & RES& SHR & \%CPU &\%MEM&Optim.&filesize&COMMAND \\
\hline
3416&1564&1416 &  0.0 & 0.2 & -O0 & 7123510 & dibbler-server\\
3416&1560&1416 &  0.0 & 0.2 & -O0 & 751948  & dibbler-server-stripped\\
3328&1544&1400 &  0.0 & 0.2 & -O0 & 6533375 & dibbler-server-wo-poslib\\
3328&1548&1400 &  0.0 & 0.2 & -O0 & 663592  & dibbler-server-wo-poslib-stripped\\
3220&1436&1292 &  0.0 & 0.2 & -Os & 4596760 & dibbler-server run\\
3140&1424&1276 &  0.0 & 0.2 & -Os & 468776  & dibbler-server-wo-poslib\\
3388&1636&1496 &  0.0 & 0.2 & -O0 & 9771605 & dibbler-client\\
3392&1644&1496 &  0.0 & 0.2 & -O0 & 725352  & dibbler-client-stripped\\
3296&1608&1472 &  0.0 & 0.2 & -O0 & 9183726 & dibbler-client-wo-poslib\\
3300&1612&1472 &  0.0 & 0.2 & -O0 & 639240  & dibbler-client-wo-poslib-stripped\\
3212&1472&1336 &  0.0 & 0.2 & -Os & 5901734 & dibbler-client-wo-poslib\\
3120&1456&1320 &  0.0 & 0.2 & -Os & 458984  & dibbler-client-wo-poslib\\
\hline
\end{tabular}
\end{center}

Dibbler stores data internally in lists. This means that server's
memory and CPU usage is a linearly proportional to a number of clients
it currently supports.

FIXME: Long/performance tests are required.

\section{Basic source code informations}
This section describes various aspects of Dibbler compilation, usage
and internal design.

\subsection{Option values and filenames}

DHCPv6 is a relatively new protocol and additional options are in a
specification phase. It means that until standarisation process is
over, they do not have any officially assigned numbers. Once
standarization process is over (and RFC document is released), this
option gets an official number. 

There's pretty good chance that different implementors may choose
diffrent values for those not-yet officialy accepted options. To
change those values in Dibbler, you have to modify file
misc/DHCPConst.h and recompile server or client. Make sure that you
build everything for scratch. Use \verb+make clean+ in Linux and
\verb+Clean up solution+ in Windows before you start building a new
version.

In default build, Dibbler stores all information in the
\verb+/var/lib/dibbler+ directory (Linux) or in the working directory
(Windows). There are multiple files stored in those
directories. However, sometimes there is a need to build Dibbler which uses
different directory or filename. To do so, simply edit
\verb+misc/Portable.h+ file and rebuild everything.

\subsection{Memory Manegement using SmartPtr}
To effectively fight memory leaks, clever mechanism was
introduced. Smart pointers are used to point to all dynamic
structures, e.g. messages, options or client informations in server
database. Smart pointer will free object by itself, when object is no
longer needed. When this is happening? When last smart pointer stops
pointing at the object. There is a tradeoff: normal pointers (*)
should not be mixed with smart pointers. 

Smart pointers are implemented as C++ class templates. Template is
called \verb+SmartPtr<TYPE>+.

To quickly explain smart pointers usage, here's short code example:
\begin{verbatim}
 1 void foo()  {
 2   SmartPtr<TIPv6Addr> addr = new TIPv6Addr("ff02::1:2");
 3   SmartPtr<TIPv6Addr> tmp;
 4   if (!tmp) cout << "Null pointer" << endl;
 5   tmp = addr;
 6   std::cout << addr->getPlain();
 7 }
\end{verbatim}
What's happened in those lines?
\begin{description}
\item[1] -- Function starts.
\item[2] -- New TIPv6Addr object is created. Smart Pointer
  (SmartPtr$<$TIPv6Addr$>$) is also created to point at this object. Using
  normal pointer to achive the same goal would look like this: \\
  \verb+TIPv6Addr * addr = new TIPv6Addr("ff02::1:2");+
\item[3] -- Another pointer is created. It is equivalent of the
  classical pointer (TIPv6Addr * tmp).
\item[4] -- Simple check if pointer does not point to anything.
\item[5] -- Smart pointers can be coppied in a easy way.
\item[6] -- Using object pointed by smart pointer is simple
\item[7] -- Here magic begins. addr and tmp are local variables, so
  they are destroyed here. But they are the only smart pointers which
  access TIPv6Addr object. So they are destroy that object. 
\end{description}

In conclusion, object remain in memory as long as there is at least
one smart pointer which points to this object. SmartPointers can be
easily derefernced. Just add * before them:
\begin{verbatim}
cout << *addr << endl;
\end{verbatim}

SmartPtrs are often used to store various objects in a list. Cool part
of this solution is that you can hold objects of various derived
classes on one list in a very comfortable manner. There is an
additional template defined to create and manipulate such lists. It is
called TContainer. There's also useful macro defined to use this
without typing too much. Here are two examples how to define list of
addresses (both mean exactly the same):
\begin{verbatim}
TContainer< SmartPtr<TIPvAddr> > addrLst;
List(TIPv6Addr) addrLst;
\end{verbatim}

How to use this list? Oh well, another example:
\begin{verbatim}
1  List(TIPv6Addr) addrLst;
2  SmartPtr<TIPv6Addr> ptr = ...;
3  SmartPtr<TIPv6Addr> tmp;
4  addrLst.clear();
5  addrLst.append(ptr);
6  addrLst.first();
7  tmp = addrLst.get();
8  cout << "List contains " << addrLst.count() << " elements" << endl;
9  addrLst.first();
10 while (tmp = addrLst.get()) 
11   cout << *tmp << endl;
\end{verbatim}

And here is description what that code does:

\begin{description}
\item[1] -- Address list declaration.
\item[2,3] -- SmartPtrs declarations. Just to show variable types.
\item[4] -- List can be cleared. All pointers will be destroyed. If
  they were only pointers to point to some objects, those objects will
  be destroyed, too.
\item[5] -- Append object pointed by ptr to the list.
\item[6] -- Rewind list to the beginning.
\item[7] -- Get next object from the list. If list is empty or last
  element was already got, NULL is returned.
\item[8] -- An easy way to count elements on the list.
\item[9] -- Rewind list to the beginning.
\item[10,11] -- A cute example how to print all addresses on the list.
\end{description}

\subsection{Logging}
To log various informations, Log(LOGLEVEL) macros are defined. There
are eight levels of logging:
\begin{description}
\item[Emergency] -- Used to report system wide emergency. Such
  conditions could not occur in the DHCPv6 client o server, so this
  logging level should not be used. Called with
  \verb+Log(Emerg) << "..." << LogEnd+.

\item[Alert] -- Used to alert an administrator about system wide
  alerts. This logging level should not be used in DHCPv6. 
  Called with \verb+Log(Alert) << "..." << LogEnd+.

\item[Critical] -- Used in situations critical to the application,
  e.g. application shutdown. Fatal errors should be logged on this
  level. Called with \verb+Log(Crit) << "..." << LogEnd+.

\item[Error] -- Used to report error situations. For example, problems
  with binding sockets. Called with \verb+Log(Error) << "..." << LogEnd+.

\item[Warning] -- Used to report RFC violations, e.g. missing required
  options, invalid parameters and so on. Called with \verb+Log(Warning) << "..." << LogEnd+.

\item[Notice] -- Used to report normal operations, e.g. address
 assignement or informations about received options. Called with
 \verb+Log(Notice) << "..." << LogEnd+.

\item[Info] -- Used to report detailed information. DHCPv6 protocol
  knowledge might be needed to understand those messages.
  Called with \verb+Log(Info) << "..." << LogEnd+.

\item[Debug] -- Used to report internal informations. Knowledge about
  Dibbler source code might be needed to understand those messages.
  Called with \verb+Log(Debug) << "..." << LogEnd+.
\end{description}

\subsection{Names and prefixes}
To avoid confussion, various prefixes are used in class and variable
names. Class types begin with T (e.g. address class would be named
TAddr), enumeration types begin with E (e.g. state enumaterion would
be names EState). Dibbler is divided into 4 large functional blocks
called managers\footnote{They are described in the following sections
  of this document}: address maganger, interface manager, Configuration
manager, and transmsission manager. Each of them uses different
prefix: Addr, Iface, Cfg or Trans. There are also objects shared among
them: messages (Msg prefix) and options (Opt prefix). Often there are
two derived versions: related to client (Clnt prefix) or related to
server (Clnt). Rel prefix is used to denote Relay related classes. 
Here are examples of some class names:
\begin{description}
\item[TAddrMgr] -- Address manager, common version.
\item[TClntAddrMgr] -- Address manager, client version.
\item[TAddrIface] -- Interface representation, used in address manager.
\item[TAddrAddr] -- Address representation used in address manager.
\item[TSrvIfaceMgr] -- Interface manager, server version.
\item[TClntIfaceIface] -- Interface representation used in client
  interface manager.
\item[TClntMsg] -- Message represented on the client side.
\item[TClntOptPreference] -- Prefernce option used on the client side.
\item[TIfaceSocket] -- Socket used in the interface manager.
\item[TClntCfgAddr] -- Address used in the client config manager.
\end{description}

Also note that class function names start with small letters
(e.g. \verb+bool TOpt::isValid();+) and class variables start with capital
letters (e.g. \verb+bool TOpt::IsValid;+).

\subsection{Configuration file parsers}
\textbf{Note:} Similar approach is used in server, client and
relay. In following section when reference to a specific file is needed,
client files are used. To find corresponding files related to server and
relay, substitute \verb+Clnt+ with \verb+Srv+ or \verb+Rel+.

Dibbler uses standard lexer/parser. Lexer is generated using flex. Parser is
generated with bison++ (full source code for bison++ is provided with
Dibbler sources). See \verb+ClntCfgMgr/ClntParser.y+ and
ClntCfgMgr/ClntLexer.l for details. Make sure that you have flex installed
(bison++ is provided with the dibbler source code). To generate parser
and lexer code, type:

\begin{verbatim}
make bison (just once, to compile bison++)
make parser (each time you modify *.l or *.y files)
\end{verbatim}

\subsubsection{Parsing}
Configuration file reading is done using Flex and bison++ tools. Flex
is so called lexer. Its responsibility is to read config file and
translate it into stream of tokens. \footnote{To be precise, Flex
  generates lexers, so it should be called lexer generator.} For
example, this config file:
\begin{verbatim}
iface eth0 {
  class { pool 2000::1-2000::9 }
}
\end{verbatim}

would be translated to following stream of tokens: [IFACE]
[STRING:eth0] [{] [CLASS] [{] [POOL] [ADDR:2000::1] [-] [ADDR:2000::9]
[}] [}]. This stream of token is then passed to parser. This parser is
generated by bison++. Parser checks if that particular sequence of
tokens makes sense. In this example, interface object will be created,
which contains one class object, which contains one pool.

Is is sometimes very useful to define some parameter, usually
associated with some level, on higher scope level. For example, if
there are 3 classes, instead of defining the same valid-lifetime value
on each of them, that parameter may be defined on the interface
level or even at the top level. This is important to remember during
parsing. Each subsequent element must inherit its parent properties
(class object must inherit parameter values defined on the interface
level).

To accomplish this feat, simple stack was implemented. For example, in
server parser, following methods are called before and after interface
definitions.

\begin{verbatim}
void SrvParser::StartIfaceDeclaration()
{
    // create new option (representing this interface) on the parser stack
    ParserOptStack.append(new TSrvParsGlobalOpt(*ParserOptStack.getLast()));
    SrvCfgAddrClassLst.clear();
}

bool SrvParser::EndIfaceDeclaration()
{
    // create and add new interface to SrvCfgMgr
    ...
    // remove last option (representing this interface) from the parser stack
    ParserOptStack.delLast();
    return true;
}   
\end{verbatim}


\subsubsection{Using parsed values}

Lexer and parser are created in the Client Configuration Manager. See
ClntCfgMgr/ClntCfgMgr.cpp. Following code is executed in the ClntCfgMgr
constructor\footnote{Actual code is much more complicated, but
unnecessary lines were removed for a clarification reasons.}

\begin{verbatim}
yyFlexLexer lexer(\&f,\&clog);
ClntParser parser(\&lexer);
result = parser.yyparse();
matchParsedSystemInterfaces(\&parser);
validateConfig();
\end{verbatim}

f and clog are normal C++ ifstream and ofstrem objects, associated with
configuration file or a standard output. Configuration file is passed to
the constructor of the entire TDHCPClient object, which is usually
located in the main() function.

Example mentioned above works as follows:

\begin{itemize}
 \item Read all interfaces from the system (using System API). This is done in
       Interface Manager and is not important right now.
 \item Create lexer object (it will read configuration file and convert it into stream
       of tokens)
 \item create parser, which will interpret stream of tokens.
 \item Match interfaces present in system with those specified in the configuration
       file.
 \item Validate configuration file to check if there are no logical errors, like T1>T2,
       specified both stateless and request for ia, etc.
\end{itemize}

\subsubsection{Embedded configuration}

\textbf{Note:} This feature applies to the client only.

Another way of defining client configuration was introduced in the 0.5.0
release. Instead of reading configuration file, configuration can be
hardcoded in the binary file itself. See MOD\_CLNT\_EMBEDDED\_CFG flag
description in section \ref{modular-features}.


%% CLIENT/SERVER/RELAY ARCHITECTURE
%%
%% Dibbler - a portable DHCPv6
%%
%% authors: Tomasz Mrugalski <thomson@klub.com.pl>
%%          Marek Senderski <msend@o2.pl>
%%
%% released under GNU GPL v2 or later licence
%%
%% $Id: dibbler-devel-arch.tex,v 1.2 2005-01-23 23:16:56 thomson Exp $
%%
%% $Log: not supported by cvs2svn $
%% Revision 1.1  2004/12/01 20:55:49  thomson
%% Documentation updated.
%%
%% Revision 1.3  2004/11/25 01:16:36  thomson
%% *** empty log message ***
%%
%% Revision 1.2  2004/10/25 20:45:54  thomson
%% Option support, parsers rewritten. ClntIfaceMgr now handles options.
%%
%% Revision 1.1  2004/07/05 01:04:40  thomson
%% Initial version.
%%
%%

\section{Common Architecture}

General architecture is common between server and client. In both
cases, all classes are divided into several major groups:
\begin{itemize}
\item[IfaceMgr] -- Interface Manager. It represents all network interfaces present in the
  system. They're represented by TIfaceIface objects and stored in
  IfaceLst. Each interface has list of open sockets, represented with
  TIfaceSocket objects. There are also a number of auxiliary functions
  for getting proper interface. IfaceIface objects  also provide
  methods to add, update and remove addresses.
\item [AddrMgr] -- Address Manager. It is an address database, which
  stores all informations about clients, IAs and associated addresses.
\item [CfgMgr] -- Config Manager. It is being used to read
  configuration information from config file and provide those
  informations while runtime. Common mechanisms shared between server
  and client are scarce, so this base class is almost empty.
\item [TransMgr] -- Transmission Manager, sometimes called Transaction
  Manager. It is responsible for network interaction and core DHCPv6
  logic. It sends various messages when such need arise, matches received
  responses with sent messages, retransmits messages etc. It contains
  list of messages currently being trasmitted. 
\item[Messages] -- There is one parent class of all messages. It
  contains several basic functionalites common to all messages.
\item[Options] -- There are multiple option classes. Note that some
  classes are designed to represent one specific option
  (e.g. OptIAAddress) and other are not (e.g. OptAddrLst can contain
  address list, so it can be used as DNS Resolvers, SIP servers o NIS
  servers option). 
\item[Misc] -- This cathegory (or rather directory) contains various
  miscellanous classes and functions.
\end{itemize}

None of those classes is used directly. Client, server and
relay uses derived classes.

They are all created within DHCPClient or DHCPServer objects in client
or server, respectively. DHCPRelay object will perform similar
function for relays.

\section{Client Architecture}

Client is represented by a DHCPClient object. It contains 4 large
managers, each with its own functions. Also messages and options are
defined:

\begin{itemize}
\item[TClntIfaceMgr] -- contains client version of the IfaceMgr. Major
  difference is a TClntIfaceIface class, an enhanced version of the
  IfaceIface. It provides methods to set up various options on the
  physical interface. Those methods are used by Options representing
  options.
\item[TClntAddrMgr] -- Client version supports additional, client
  related functions, e.g. tentative timeout used in DAD procedure. It
  also simplifies database handling as there will always be only one
  client in the database.
\item[TClntCfgMgr] -- Client related parser. TClntCfgMgr and related
  objects are designed to provide easy access to parameters specified
  in the configuration file. ClntCfgIface is a very important class as
  most of the parameters is interface-specific.
\item[TClntTransMgr] -- Core logic of the Client. It uses all other
  managers to decide what actions should be taken at occuring
  circumstances, e.g. send REQUEST when there are less addresses
  assigned than specified in the configuration file.
\item[TClntMsg] -- All messages have client specific
  classes. Those objects are created as new messages are being
  sent. After server message reception, object is also created and
  passed to the original message. For example, client sends
  \msg{SOLICIT} message and server send \msg{ADVERTISE} message. Reply
  will be passed by invoking \verb+answer(msgAdvertise)+ method on the
  \verb+msgSolicit+ object.
\item[TClntOpt] -- There are client specific options defined. Each
  of those options has \verb+doDuties()+ method which is called if this
  option was received in a proper reply message from the server. It
  calls appropriate methods in TClntIfagrMgr which set specific options
  in the system.
\end{itemize}

\section{Server Architecture}

Server is represented by a DHCPServer object. It contains 4 large
managers, each with its own functions. Also SrvMessages and SrvOptions
are defined:
\begin{itemize}
\item[TSrvIfaceMgr] -- contains server version of the IfaceMgr. There
  are almost no modificiation compared to common version.

\item[TSrvAddrMgr] -- Client version supports additional, client
  related functions, e.g. tentative timeout used in DAD procedure. It
  also simplifies database handling as there will always be only one
  client in the database.
\item[TSrvCfgMgr] -- Client related parser. TSrvCfgMgr and related
  objects are designed to provide easy access to parameters specified
  in the configuration file. SrvCfgIface is a very important class as
  most of the parameters is interface-specific.
\item[TSrvTransMgr] -- Core logic of the client. It uses all other
  managers to decide what actions should be taken at occuring
  circumstances, e.g. send REQUEST when there are less addresses
  assigned than specified in the configuration file.
\item[TSrvMsg] -- Server version of the messages. Each time server
  receives a message, TSrvMsg is created. Depending of its type,
  TSrvAdvertise of TSrvReply message is created. As parameter to its
  contructor original message is passed. After creating message, it is
  sent back to the client and stored for possible retransmission
  purposes.
\item[TSrvOpt] -- Server version of the Option representing
  objects. They are just used to store data, so they are considerably
  simpler than client versions.
\end{itemize}

\section{Relay Architecture}
Preliminary relay version was available in the 0.4.0 release. It
consists of serveral simple blocks:
\begin{itemize}
\item[TRelIfaceMgr] -- contains relayr version of the IfaceMgr. There
  are almost no modificiation compared to common version, execept
  decodeMsg() and decodeRelayRepl() methods.
\item[TRelCfgMgr] -- Relay related parser. TRelCfgMgr and related
  objects are designed to provide easy access to parameters specified
  in the configuration file. RelCfgIface is a very important class as
  most of the parameters is interface-specific.
\item[TRelTransMgr] -- It's plain simple manager. It's only function
  is to relay received message on all interfaces.
\item[TRelMsg] -- From the relay's point of view, all messages fall to
  one of 3 categories: Generic (i.e. not encapsulated) messages,
  RelayForw (already forwarded by some other relay) and RelayRepl
  (replies from server). Most of the messages is threated as generic
  message.
\item[TRelOpt] -- Similar approach is used to handle options. Expect
  RELAY\_MSG option (which contains relayed message) and interface-id
  option (which contains identifier of the interface), all options are
  threated as generic options, which are handled transparently.
\end{itemize}


%% HISTORY, CONTACT
%%
%% Dibbler - a portable DHCPv6
%%
%% authors: Tomasz Mrugalski <thomson@klub.com.pl>
%%          Marek Senderski <msend@o2.pl>
%%
%% released under GNU GPL v2 or later licence
%%
%% $Id: dibbler-devel-misc.tex,v 1.2 2004-12-01 20:55:49 thomson Exp $
%%
%% $Log: not supported by cvs2svn $
%% Revision 1.1  2004/07/05 01:04:40  thomson
%% Initial version.
%%
%%

\section{Miscellanous tips}

\begin{itemize}
\item Ethereal, a widely used network sniffer/analyzer has a bug with
  parsing DHCPv6 message: SIP options are always reported as
  malformed. Also NIS/NIS+ options have improper values (not
  comformant to RFC3898). To work around that problem, download
  packet-dhcpv6.c from Dibbler homepage and recompile Ethereal.
\item If you are reading this Developer's Guide, then Hey! You're
  probably a developer! If you found any bugs (or think you found
  one), go to the
  \href{http://klub.com.pl/bugzilla}{http://klub.com.pl/bugzilla}
   and report it. If you report was a mistake -- oh well, you just
  lost 5 minutes. But if it was really a bug, you have just improved
  next Dibbler version.
\item If you have any questions about Dibbler or DHCPv6, feel free to
  mail me.
\end{itemize}


\end{document}